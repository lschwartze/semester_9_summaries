\documentclass[a4paper, 12pt]{article}

\usepackage{fullpage}
\usepackage[utf8]{inputenc}
\usepackage[english]{babel}
\usepackage{amsmath,amssymb}
\usepackage[explicit]{titlesec}
\usepackage{ulem}
\usepackage[onehalfspacing]{setspace}
\usepackage{amsthm}

\theoremstyle{plain}
\newtheorem{theorem}{Satz}[section] % reset theorem numbering for each chapter

\theoremstyle{definition}
\newtheorem{definition}[theorem]{Definition} % definition numbers are dependent on theorem numbers
\theoremstyle{lemma}
\newtheorem{lemma}[theorem]{Lemma}

\theoremstyle{remark}
\newtheorem{remark}[theorem]{Bemerkung}

\theoremstyle{corollary}
\newtheorem{corollary}[theorem]{Korollar}

\theoremstyle{example}
\newtheorem{example}[theorem]{Beispiel}

\titleformat{\subsection}
{\small}{\thesubsection}{1em}{\uline{#1}}
\begin{document}
	\begin{titlepage} 
		\title{Topologische Flächen und Fundamentalgruppen Zusammenfassung}
		\clearpage\maketitle
		\thispagestyle{empty}
	\end{titlepage}
	\tableofcontents
	\newpage
	\section{Topologische Flächen}
	\subsection{Einführung}
	\begin{definition}[Mannigfaltigkeit]
		Sei $n \in \mathbb{N}$. Eine $n$-Mannigfaltigkeit ist ein topologischer Raum $X$ sodass \begin{enumerate}
			\item $X$ ist Hausdorff'sch
			\item die Topologie besitzt eine abzählbare Basis
			\item jeder Punkt $x \in X$ besitzt eine Umgebung $x\in U\subseteq X$, die homöomorph zu einer offenen Teilmenge $V\subseteq \mathbb{R}^n$ ist. Ein Homöomorphismus \[\varphi: \, U \tilde{\rightarrow} V \subseteq \mathbb{R}^n\] heißt Karte.
			\item $X$ ist zusammenhängend
		\end{enumerate}
	\end{definition}
	Für $n=1$ heißt $X$ eine Kurve, für $n=2$ eine Fläche.
	\subsection{Klassifikation der Kurve}
	\begin{theorem}
		Jede Kurve ist homöomorph zu genau einer der folgenden Kurven \begin{enumerate}
			\item $\mathbb{R}$
			\item $S^1$
		\end{enumerate}
	\end{theorem}
	\begin{example}
		Sei $X = \{(x,y) \in \mathbb{C}^2 : \; y^2 = x^3 - x\}$. Das wichtigste Hilfsmittel, um die Topologie von $X$ zu verstehen, ist die Projektion \[\pi: X \to \mathbb{C}\] mit \[\pi(x,y) = x\] Für $a \in {0, \pm 1}$ hat $a$ genau ein Urbild, ansonsten 2.
	\end{example}
	\begin{definition}
		Eine stetige Abbildung $\pi:Y \to X$ heißt Überlagerung, wenn jeder Punkt $x \in X$ eine offene Umgebung $U \subseteq X$ besitzt, sodass \[\pi^{-1}(U) = \bigcup_{i \in I} V_i\] \[\pi|_{V_i} : V_i \tilde{\rightarrow} U\] ein Homöomorphismus $\forall i$. 
	\end{definition}
	\begin{definition}
		Sei $K$ ein Körper, $n \in \mathbb{N}$. Sei \[\mathbb{P}^n(K) = K^{n+1}\setminus\{(0,0,...,0)\}/\sim\]
		mit \[(z_0,...,z_n) \sim (z_0',...,z_n')\] genau dann, wenn \[\exists t \in K^* : \; z_i' = tz_i\]
	\end{definition}
	\begin{example}
		$\mathbb{P}^1(\mathbb{C}) = \{[z_0:z_1] \in \mathbb{C}^2 \setminus \{(0,0)\}\} \tilde{=} \mathbb{C} \cup \{\infty\}$ durch die Bijektion \[[z:1] \leftarrow z\]
		\[[1:0] \leftarrow \infty\]
	\end{example}
	\subsection{Klassifizierung der kompakten Flächen}
	\begin{table}[h]
		\centering
		\begin{tabular}{c | c | c}
			$g$ & orientierbar & nicht orientierbar\\
			\hline \hline
			0 & $S^2$ & $\mathbb{P}^2(\mathbb{R})$\\
			1 & Torus & Klein'sche Flasche\\
			2 & Doppeltorus & \vdots\\
			3 & Tripeltorus & \vdots
		\end{tabular}
	\end{table}
\end{document}