\documentclass[a4paper, 12pt]{article}

\usepackage{fullpage}
\usepackage[utf8]{inputenc}
\usepackage[english]{babel}
\usepackage{amsmath,amssymb}
\usepackage[explicit]{titlesec}
\usepackage{ulem}
\usepackage[onehalfspacing]{setspace}
\usepackage{amsthm}

\theoremstyle{plain}
\newtheorem{theorem}{Satz}[section] % reset theorem numbering for each chapter

\theoremstyle{definition}
\newtheorem{definition}[theorem]{Definition} % definition numbers are dependent on theorem numbers
\theoremstyle{lemma}
\newtheorem{lemma}[theorem]{Lemma}

\theoremstyle{remark}
\newtheorem{remark}[theorem]{Bemerkung}

\theoremstyle{corollary}
\newtheorem{corollary}[theorem]{Korollar}

\theoremstyle{example}
\newtheorem{example}[theorem]{Beispiel}

\titleformat{\subsection}
{\small}{\thesubsection}{1em}{\uline{#1}}
\begin{document}
	\begin{titlepage} 
		\title{Topologische Flächen und Fundamentalgruppen Zusammenfassung}
		\clearpage\maketitle
		\thispagestyle{empty}
	\end{titlepage}
	\tableofcontents
	\newpage
	\section{Topologische Flächen}
	\subsection{Einführung}
	\begin{definition}[Mannigfaltigkeit]
		Sei $n \in \mathbb{N}$. Eine $n$-Mannigfaltigkeit ist ein topologischer Raum $X$ sodass \begin{enumerate}
			\item $X$ ist Hausdorff'sch
			\item die Topologie besitzt eine abzählbare Basis
			\item jeder Punkt $x \in X$ besitzt eine Umgebung $x\in U\subseteq X$, die homöomorph zu einer offenen Teilmenge $V\subseteq \mathbb{R}^n$ ist. Ein Homöomorphismus \[\varphi: \, U \tilde{\rightarrow} V \subseteq \mathbb{R}^n\] heißt Karte.
			\item $X$ ist zusammenhängend
		\end{enumerate}
	\end{definition}
	Für $n=1$ heißt $X$ eine Kurve, für $n=2$ eine Fläche.
	\subsection{Klassifikation der Kurve}
	\begin{theorem}
		Jede Kurve ist homöomorph zu genau einer der folgenden Kurven \begin{enumerate}
			\item $\mathbb{R}$
			\item $S^1$
		\end{enumerate}
	\end{theorem}
	\begin{example}
		Sei $X = \{(x,y) \in \mathbb{C}^2 : \; y^2 = x^3 - x\}$. Das wichtigste Hilfsmittel, um die Topologie von $X$ zu verstehen, ist die Projektion \[\pi: X \to \mathbb{C}\] mit \[\pi(x,y) = x\] Für $a \in {0, \pm 1}$ hat $a$ genau ein Urbild, ansonsten 2.
	\end{example}
	\begin{definition}
		Eine stetige Abbildung $\pi:Y \to X$ heißt Überlagerung, wenn jeder Punkt $x \in X$ eine offene Umgebung $U \subseteq X$ besitzt, sodass \[\pi^{-1}(U) = \bigcup_{i \in I} V_i\] \[\pi|_{V_i} : V_i \tilde{\rightarrow} U\] ein Homöomorphismus $\forall i$. 
	\end{definition}
	\begin{definition}
		Sei $K$ ein Körper, $n \in \mathbb{N}$. Sei \[\mathbb{P}^n(K) = K^{n+1}\setminus\{(0,0,...,0)\}/\sim\]
		mit \[(z_0,...,z_n) \sim (z_0',...,z_n')\] genau dann, wenn \[\exists t \in K^* : \; z_i' = tz_i\]
	\end{definition}
	\begin{example}
		$\mathbb{P}^1(\mathbb{C}) = \{[z_0:z_1] \in \mathbb{C}^2 \setminus \{(0,0)\}\} \tilde{=} \mathbb{C} \cup \{\infty\}$ durch die Bijektion \[[z:1] \leftarrow z\]
		\[[1:0] \leftarrow \infty\]
	\end{example}
	\section{Klassifizierung der kompakten Flächen}
	\begin{table}[h]
		\centering
		\begin{tabular}{c | c | c}
			$g$ & orientierbar & nicht orientierbar\\
			\hline \hline
			0 & $S^2$ & $\mathbb{P}^2(\mathbb{R})$\\
			1 & Torus & Klein'sche Flasche\\
			2 & Doppeltorus & \vdots\\
			3 & Tripeltorus & \vdots
		\end{tabular}
	\end{table}
	\subsection{Triangulierung}
	\begin{definition}
		Sei $\mathbb{A}$ ein reeller Vektorraum.
		\begin{enumerate}
			\item $v_0,...,v_n \in \mathbb{A}$ heißt \underline{affin unabhängig}, wenn $v_1-v_0, ..., v_n - v_0$ linear unabhängig sind
			\item Für $v_0,...,v_n \in \mathbb{A}$ affin unabhängig heißt \[\sigma = [v_0...,v_n] = \{t_0v_0 + \dots t_n v_n \mid t_i \geq 0, \; t_0 + \dots t_n = 1\}\] der von den $v_i$ aufgespannte \underline{Simplex} der Dimension $\dim(\sigma) = n$
			\item ist $\{v_{i_0}, ..., v_{i,k}\} \subseteq \{v_0,...,v_n\}$ eine Teilmenge mit $k+1$ Elementen, dann heißt das davon erzeugte $k$-Simplex eine \underline{$k$-Seite} von $\sigma$
			\item $\partial \sigma = \bigcup_{\delta \subsetneqq \sigma} \delta$ heißt \underline{Rand} von $\sigma$ und $\overset{\circ}{\sigma} = \sigma \setminus \partial \sigma$ ist das \underline{Innere}
		\end{enumerate}
	\end{definition}
	\begin{definition}
		Ein \underline{abstrakter Simplicialkomplex} ist ein Paar $K = (V, S)$, wobei $V \neq \varnothing$ und $S$ eine Menge von endlichen Teilmengen von $V$. Anschaulich ist $K$ ein Graph mit $V$ der Menge der Ecken von $K$ und $S$ die Menge der Simplizes von $K$.\\
		$S$ muss folgende Bedingungen erfüllen: \begin{enumerate}
			\item jede Ecke $v \in V$ liegt in mindestens einem und höchstens endlich vielen Simplizes
			\item $s \in S$ und $s'\subseteq s$, dann ist $s' \in S$
		\end{enumerate}
	\end{definition}
	Sei $K = (V,S)$ ein Simplicialkomplex. Sei $$\mathbb{A} = \mathbb{R}^{\left|V\right|} = \{(t_v)_{v \in V} \in \mathbb{R}^{\left|V\right|} \mid t_v = 0 \text{ für alle bis auf endliche viele } v \in V\}$$ und $$\left|K\right| = \{(t_v)_{v \in V} \in \mathbb{A} \mid t_v \geq 0 \; \land \sum_{v \in V} t_v = 1 \land \; s = \{v \in V \mid t_v > 0\} \in S\}$$
	\begin{definition}
		$\left|K\right|$ heißt die geometrische Realisierung von $K$. Für $s \in S$ heißt $\sigma = \left|s\right| = \{(t_v)_{v \in V} \mid t_v = 0 \; \forall v \notin s\} = [v_0,...,v_n]$ die geometrische Realisierung von $s$. 
	\end{definition}
	\begin{definition}
		Basis der Topologie sind Mengen $U \subseteq \left|K\right|$ der Form \begin{itemize}
			\item $U \cap \left|s\right| \subseteq \left|s\right|$ offen für alle $s \in S$
			\item $U \cap \left|s\right| \neq \varnothing$ für alle $s$ bis auf endlich viele
		\end{itemize}
	\end{definition}
	\begin{lemma}
		Die Topologie hat folgende Eigenschaften
		\begin{enumerate}
			\item Für $v \in V$ heißt $st(v) = \bigcup_{s \in S, \, v \in s}  \overset{\circ}{\left|s\right|}$ der Stern von $v$. Das ist eine offene Umgebung von $v$ mit Abschluss $\overline{st(v)} = \bigcup_{s \in S, \; v \in S} \left|s\right|$
			\item $(st(v))_{v \in V}$ bilden eine offene Überdeckung von $\left|K\right|$ 
			\item $\overline{st(v)}$ ist kompakt, wegzusammenhängend 
			\item $\left|K\right|$ ist lokal kompakt, lokal wegzusammenhängend und Hausdorffsch
			\item $\left|K\right|$ ist zusammenhängend $\Leftrightarrow$ $\left|K\right|$ ist wegzusammenhängend $\Leftrightarrow$ $K$ zusammenhängend $\Rightarrow$ $V$ ist abzählbar und $\left|K\right|$ ist \textit{second countable}.
		\end{enumerate}
	\end{lemma}
	\begin{definition}
		Für $v \in V$ definiert $L_K(v) = (V_v, S_v)$ einen \underline{Link} von $v$ mit \[S_v = \{s\setminus v \mid s \in S : \; v \in s\}\]
	\end{definition}
	\begin{theorem}
		Sei $K$ ein zusammenhängender Simlicialkomplex, $n \geq 1$. Dann ist $\left|K\right|$ eine $n$-Mannigfaltigkeit genau dann, wenn \begin{enumerate}
			\item alle maximalen Simplizes haben Dimension $n$ ($K$ ist von reiner Dimension $n$)
			\item jeder $(n-1)$-Simplex ist eine Seite von genau zwei $n$-Simplizes
			\item $\left|L_K(v)\right| \cong S^{n-1}$
		\end{enumerate}
	\end{theorem}
	\begin{definition}
		Sei $X$ ein topologischer Raum. Eine \underline{Triangulierung} von $X$ ist ein Homöomorphismus $\left|K\right| \cong X$ wobei $K$ ein Simplicialkomplex ist.
	\end{definition}
	\begin{theorem}
		Jede Fläche ist triangulierbar.
	\end{theorem}
	\begin{remark}
		Die Aussage ist noch wahr für $n=3$ aber falsch ab $n=4$.
	\end{remark}
	\subsection{Zellkomplexe}
	\begin{definition}
		Sei $A$ eine Menge. \begin{enumerate}
			\item Die Menge der orientierten Elemente von $A$ ist \[\tilde{A} = A \cup A^{-1}\] wobei $A^{-1} = \{a^{-1} \mid a \in A\}$
			\item Die Menge der orientierten Zyklen in $A$ ist \[A^* = \{[a_1,...,a_n] \mid n \geq 0, \; a_i \in \tilde{A}\}\] wobei die Äquivalenzklassen definiert sind durch \[(a_1,...,a_n) \sim (a_1',...,a_n') \Leftrightarrow \; \exists k \geq 0: \; a_i' = \begin{cases}
				a_{i+k},& i+k \leq n\\
				a_{i+k-n},& k > n
			\end{cases}\]
		\end{enumerate}
	\end{definition}
	\begin{definition}
		Ein Zellenkomplex ist ein Tripel $K = (F,E,\delta)$, wobei $F$ (Menge der Flächen) nicht leer und endlich, $E$ (Menge der Kanten) endlich und $\delta: \tilde{F} \to E^*$ die Randabbildung sodass \begin{enumerate}
			\item $\delta(A^{-1}) = \delta(A)^{-1} \; \forall A \in F$
			\item $A_1,A_2 \in F$, $A_1 \neq A_2$, dann $\partial(A_1) \neq \partial(A_2)$
			\item jedes $a \in \tilde{E}$ kommt in genau einem oder genau zwei Rändern $\partial(A)$, $A \in \tilde{F}$ vor.
			\item $K$ ist zusammenhängend.
		\end{enumerate}
	\end{definition}
	\begin{definition}
		Sei $K = (F,E,\delta)$ ein Zellenkomplex. Wir definieren die \underline{geometrische Realisierung} von $K$ als den Quotientenraum \[\left|K\right| = (\bigcup_{A \in F} \left|A\right|)/\sim\] wobei $\left|A\right| = \cong D = \{x \in \mathbb{R}^2 \mid ||x|| \leq 1\}$ und $\sim$ wie folgt definiert ist.\\
		Ein Punkt $x \in \overset{\circ}{\left|A\right|}$ ist nur zu sich selbst äquivalent. Für $x \in \partial(\left|A\right|) \cong S^1$: Wir unterteilen $\partial(\left|A\right|) = S^1$ in $\underline{n}$ Segmente/Intervalle (wobei $\partial A = [a_1,...,a_{\underline{n}}])$, markieren das $i$-te Segment mit $a_i$ und identifizieren Punkte auf dem Segment mit $a_i$ markierten Segment mit Punkten auf jeden mit $a_i$ oder $a_i^{-1}$ markierten Segment von Rand von $\left|A_j\right|$ gemäß der Orientierung.
	\end{definition}
	\begin{theorem}
		Für jeden Zellenkomplex $K$ ist $\left|K\right|$ eine kompakte Fläche \underline{mit Rand}. 
	\end{theorem}
	\begin{definition}
		Sei $K = (F,E,\delta)$ ein Zellenkomplex und $a \in \tilde{E}$. \begin{enumerate}
			\item Ein \underline{Nachfolger} von $a$ ist ein $b \in \tilde{E}$, sodass ab in einem Rand $\partial(A)$, $A \in \tilde{F}$ vorkommt.
			\item $a$ heißt \underline{innere Kante}, falls $a$ in genau zwei Rändern vorkommt.
			\item $a$ heißt \underline{äußere Kante}, falls $a$ in genau einem Rand vorkommt.
			\item Die Relation $\sim$ ist eine Relation auf $\tilde{E}$ definiert als $a\sim b$ $\Leftrightarrow b^{-1}$ ist ein Nachfolger von $a$
			\item Sei $\approx$ die gröbste Äquivalenzrelation mit $a \sim b \Rightarrow a \approx b$
			\item Eine Ecke von $K$ ist eine Äquivalenzklasse $v = {a_1,...,a_n}$ von $\approx$, $v \in V = \tilde{E}/\approx$.
		\end{enumerate}
	\end{definition}
	\begin{lemma}
		Sei $\bar{x} \in \left|K\right|$ ein Eckpunkt, $a_1,...,a_m \in \tilde{E}$ die orientierten Kanten mit Endpunkt $\bar{x}$. \begin{enumerate}
			\item Angenommen $a_1,...,a_m$ sind innere Kanten. Dann gibt es eine zyklische Anordnung sodass $\forall i$ gilt $a_{i-1}^{-1}$ und $a_{i+1}^{-1}$ sind Nachfolger von $a$. 
			\item Sonst gibt es eine Ordnung $a_1,...,a_m$, sodass $a_1, a_m$ äußere und $a_2,...,a_{m-1}$ innere Kanten sind. Für $i=2,...,m-1$ gilt die gleiche Aussage wie oben und für $i=1,m$ jeweils modulo $m$.
		\end{enumerate}
	\end{lemma}
	\begin{definition}
		Die beiden Fälle des vorherigen Lemmas definieren \underline{innere} und \underline{äußere} Ecken.
	\end{definition}
	\begin{remark}
		$\left|K\right|$ ist eine triangulierte Fläche mit Rand $\Leftrightarrow$
		\begin{enumerate}
			\item $a,b \in \tilde{E}$, $a \sim b \Rightarrow a^{-1} \not \sim b^{-1}$.
			\item $\forall A \in \tilde{F}$: $\partial A = [a_1,a_2,a_3]$ 
		\end{enumerate}
	\end{remark}
	\begin{theorem}
		Jede kompakte Fläche mit Rand $X$ besitzt eine Darstellung $X \cong \left|K\right|$ für einen Zellkomplex $K$.
	\end{theorem}
	Es stellt sich nun die Frage, wie man für zwei Zellkomplexe $K_1,K_2$ entscheiden kann, ob $\left|K_1\right| \cong \left|K_2\right|$.\\
	\underline{Schritte}: \begin{enumerate}
		\item Definiere eine kombinatorische Äquivalenzrelation auf der Menge der Zellkomplexe, sodass $K_1\sim K_2 \Rightarrow\left|K_1\right| \cong \left|K_2\right|$
		\item Finde eine Liste von Zellkomplexen $K_1,K_2,...$, sodass für jeden Zellkomplex $K$ gilt $\left|K\right| \sim \left|K_i\right|$ für genau ein $i$.
		\item zeige $\left|K_i\right| \not \cong \left|K_j\right|$
	\end{enumerate}
	\begin{definition}
		Seine $K,K'$ Zellkomplexe. Dann heißt $K'$ eine \underline{elementare Verkleinerung} von $K$, wenn $K'$ aus $K$ durch eine Reihe von den folgenden Operationen entsteht
		\begin{enumerate}
			\item Unterteilung einer Randkante $a$ in zwei Kanten $b,c$ mit gleicher Orientierung
			\item Ergänzung einer inneren Kante durch eine Fläche $A$
		\end{enumerate} 
	\end{definition}
	Nun definiert man $\sim$ auf der Menge der Isomorphie-Klassen der Zellkomplexe durch \begin{definition}
		$\sim$ ist die gröbste Äquivalenzrelation mit $K'$ ist eine elementare Verfeinerung von $K$ $\Rightarrow$ $K' \sim K$
	\end{definition}
	\begin{lemma}
		Ist $K'$ eine elementare Verfeinerung von $K$, so gilt $\left|K'\right| \cong \left|K\right|$.
	\end{lemma}
	Für die weiteren Überlegungen müssen wir uns Invarianten von $K$ überlegen, die unter den beiden Operationen erhalten bleiben. Ab diesem Punkt wollen wir die Annahme treffen, dass alle Kanten innere Kanten sind. Das heißt, jedes $a \in \tilde{E}$ kommt genau zwei mal im Rand vor. Damit ist $\left|K\right|$ eine kompakte Fläche ohne Rand.
	\begin{definition}
		Sei $K = (F,E,\partial)$ ein Zellkomplex. Eine Orientierung von $K$ ist eine Teilmenge $O\subset \tilde{F}$, sodass
		\begin{enumerate}
			\item $\tilde{F} = O \cup O^{-1}$
			\item jedes $a \in \tilde{E}$ kommt genau ein Mal in den Rändern $\partial(A)$, $A \in O$ vor
		\end{enumerate}
	\end{definition}
	\begin{definition}
		$K=(F,E,\partial)$ heißt orientierbar, wenn eine Orientierung existiert.
	\end{definition}
	\begin{lemma}
		Sei $K'$ eine elementare Verfeinerung von $K$. Dann ist $K$ orientierbar, genau dann, wenn $K'$ orientierbar ist.
	\end{lemma}
	\begin{corollary}
		$K_1 \sim K_2 \Rightarrow$ $K_1$ ist orientierbar, genau dann, wenn $K_2$ ist orientierbar.
	\end{corollary}
	\begin{definition}
		Die Euler-Charakteristik von $K$ ist \[\chi(K) = \left|F\right| - \left|E\right| + \left|V\right|\]
	\end{definition}
	\begin{lemma}
		Sei $K_1 \sim K_2$. Dann ist $\chi(K_1) = \chi(K_2)$.
	\end{lemma}
	\begin{definition}
		Für Zellkomplexe werden zwei Normalformen definiert.
		\begin{enumerate}
			\item $F = \{A\}$, $E = \{a_1,b_1,...,a_p,b_p\}$, $p \geq 0$ mit $$\partial(A) = a_1b_1a_1^{-1}b_1^{-1}...a_pb_pa_p^{-1}b_p^{-1}$$
			\item $F = \{A\}$, $E = \{a_1,...,a_p\}$, $p \geq 1$ mit $$\partial(A) = a_1a_1...a_pa_p$$
		\end{enumerate}
	\end{definition}
	\begin{remark}
		\begin{itemize}
			\item Ist in Fall 1. $p=0$, dann ist $\left|K\right| \cong S^1$.
			\item Fall 1. ist orientierbar, Fall 2. nicht.
			\item In Fall 1. ist $\chi(K) = 2-2p$, in Fall 2. ist $\chi(K) = 2-p$.
		\end{itemize}
	\end{remark}
	\begin{corollary}
		Die Zellkomplexe in Normalform sind paarweise nicht äquivalent.
	\end{corollary}
	Man kann sich überlegen, dass die beschriebene Ebene in Fall 2 ein $p$-Torus ist. Deswegen wird der Teil $aba^{-1}b^{-1}$ des Randes eines Zellkomplexes auch Henkel. Analog nennt man die Teile des Randes einer Fläche in zweiter Normalform eine Kreuzhaube.
	\begin{theorem}
		Jeder Zellenkomplex ist äquivalent zu genau einem Zellkomplex in Normalform.
	\end{theorem}
	\begin{corollary}
		Jede kompakte Fläche ist homöomorph zu einer der Flächen in Normalform ($\left|K\right|, K$ in NF), d.h. eine zusammenhängede Summe von $p \geq 0$ Henkeln oder $p \geq 1$ Kreuzhauben.
	\end{corollary}
	\begin{theorem}
		Sind $K_1,K_2$ in verschiedenen Normalformen, dann ist $\left|K_1\right| \not \cong \left|K_2\right|$.
	\end{theorem}
	Die Aussage ist eine Konsequenz des folgenden Lemmas.
	\begin{lemma}
		Orientierbarkeit und Euler-Charakteristik sind topologische Invarianten.
	\end{lemma}
	\underline{INTERMEZZO}: Dafür kann die sogenannte singuläre Homologie verwendet werden.
	\begin{definition}
		Sei $X$ ein topologischer Raum. Die \underline{singulären Homologiegruppen} $H_q(x)$, $q\geq 0$ sind alle abelschen Gruppen, die funktionell von $X$ abhängen. D.h. für $f:X\to Y$ enthält man abelsche Gruppen $f_{\star, q}: H_q(x) \to H_q(y)$ sodass \[(I_x)_x = I_{H_q(x)}\] und \[(f\circ g)_\star = f_\star \circ g_\star\]
		Sei $X$ eine kompakte Fläche, dann ist
		\begin{enumerate}
			\item $H_0(x) \cong \mathbb{Z}$ (allgemein: $H_0$ ist freie abelsche Gruppe mit Rang gleich der Anzahl der Wegzusammenhangskomponenten)
			\item $H_1(x) = \frac{\langle\gamma \mid \gamma: S^1 \to \to X\rangle_\mathbb{Z}}{\langle\partial A \mid A \subseteq X\rangle}$
		\end{enumerate}
	\end{definition}
	\begin{lemma}
		Ist allgemein $X$ eine kompakte Fläche in Normalform, so ist \begin{enumerate}
			\item $H_0(x) \cong \mathbb{Z}$
			\item $H_1(x) \cong \begin{cases}
				\mathbb{Z}^{2p}, & (I), p \geq 0\\
				\left(\mathbb{Z}/2\mathbb{Z}\right)^{p}, & (II), p \geq 1
				\end{cases}$
			\item $H_2(x) = \begin{cases}
				\mathbb{Z}, & (I)\\
				0, & (II)
				\end{cases}$
		\end{enumerate}
	\end{lemma}
	Alternativ lässt sich das Lemma mittels der Fundamentalgruppe beweisen.
\end{document}