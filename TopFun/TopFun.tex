\documentclass[a4paper, 12pt]{article}

\usepackage{fullpage}
\usepackage[utf8]{inputenc}
\usepackage[english]{babel}
\usepackage{amsmath,amssymb}
\usepackage[explicit]{titlesec}
\usepackage{ulem}
\usepackage[onehalfspacing]{setspace}
\usepackage{amsthm}

\theoremstyle{plain}
\newtheorem{theorem}{Satz}[section] % reset theorem numbering for each chapter

\theoremstyle{definition}
\newtheorem{definition}[theorem]{Definition} % definition numbers are dependent on theorem numbers
\theoremstyle{lemma}
\newtheorem{lemma}[theorem]{Lemma}

\theoremstyle{remark}
\newtheorem{remark}[theorem]{Bemerkung}

\theoremstyle{corollary}
\newtheorem{corollary}[theorem]{Korollar}

\theoremstyle{example}
\newtheorem{example}[theorem]{Beispirl}

\titleformat{\subsection}
{\small}{\thesubsection}{1em}{\uline{#1}}
\begin{document}
	\begin{titlepage} 
		\title{Topologische Flächen und Fundamentalgruppen Zusammenfassung}
		\clearpage\maketitle
		\thispagestyle{empty}
	\end{titlepage}
	\tableofcontents
	\newpage
	\section{Topologische Flächen}
	\subsection{Einführung}
	\begin{definition}[Mannigfaltigkeit]
		Sei $n \in \mathbb{N}$. Eine $n$-Mannigfaltigkeit ist ein topologischer Raum $X$ sodass \begin{enumerate}
			\item $X$ ist Hausdorff'sch
			\item die Topologie besitzt eine abzählbare Basis
			\item jeder Punkt $x \in X$ besitzt eine Umgebung $x\in U\subseteq X$, die homöomorph zu einer offenen Teilmenge $V\subseteq \mathbb{R}^n$ ist. Ein Homöomorphismus \[\varphi: \, U \tilde{\rightarrow} V \subseteq \mathbb{R}^n\] heißt Karte.
			\item $X$ ist zusammenhängend
		\end{enumerate}
	\end{definition}
	Für $n=1$ heißt $X$ eine Kurve, für $n=2$ eine Fläche.
	\subsection{Klassifikation der Kurve}
	\begin{theorem}
		Jede Kurve ist homöomorph zu genau einer der folgenden Kurven \begin{enumerate}
			\item $\mathbb{R}$
			\item $S^1$
		\end{enumerate}
	\end{theorem}
\end{document}