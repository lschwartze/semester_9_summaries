\documentclass[a4paper, 12pt]{article}

\usepackage{fullpage}
\usepackage[utf8]{inputenc}
\usepackage[english]{babel}
\usepackage{amsmath,amssymb}
\usepackage[explicit]{titlesec}
\usepackage{ulem}
\usepackage[onehalfspacing]{setspace}
\usepackage{amsthm}

\theoremstyle{plain}
\newtheorem{theorem}{Theorem}[section] % reset theorem numbering for each chapter

\theoremstyle{definition}
\newtheorem{definition}[theorem]{Definition} % definition numbers are dependent on theorem numbers
\theoremstyle{lemma}
\newtheorem{lemma}[theorem]{Lemma}

\theoremstyle{remark}
\newtheorem{remark}[theorem]{Remark}

\theoremstyle{corollary}
\newtheorem{corollary}[theorem]{Corollary}

\theoremstyle{example}
\newtheorem{example}[theorem]{Example}

\titleformat{\subsection}
{\small}{\thesubsection}{1em}{\uline{#1}}
\begin{document}
	\begin{titlepage} 
		\title{Opiii}
		\clearpage\maketitle
		\thispagestyle{empty}
	\end{titlepage}
	\tableofcontents
	\newpage
	\section{Dynamic Networks}
	Dynamic graph networks are graph networks that change over time. Communication is in synchronous, asynchronous or semi-synchronous rounds. Additionally shared memory is possible. Network elements may be failure-free or failure-prone. A classical example are \underline{mobile ad-hoc networks}. Those are temporary interconnection networks of mobile wireless nodes without a fixed infrastructure. Communication happens whenever mobile nodes come within the wireless range of each other.
	\begin{example}
		In mobile ad hoc networks, one may want to colour the graph or maintain a routing mechanism for communication to any particular destination in the network.
	\end{example}
	\subsection{Almost constant message-passing vertex colouring in a tree}
	Let $T$ be a tree network with $n$ labelled vertices in $[n]$. Colouring the graph can be done in almost constant, i.e. in $\log^*$ time.
	\begin{definition}
		$\log^*(x)$ is defined as the number of $\log$ functions that need to be applied to $x$ such that the result is at most 1. E.g. $\log^*(16) = 3$ and $\log^*2^{65536} = 5$.
	\end{definition}
	\begin{enumerate}
		\item begin by rooting the tree at vertex $0$. This defines an order on the tree
		\item each parent sends its number to all of its children
		\item each child computes the smallest index $i$ where its number differs from the parent's number. It is important to note that this can be done in constant time with suitable hardware
		\item It computes a new ID for itself consisting of a trailing bit corresponding to the bit where IDs disagreed. The new ID begins with the binary representation of the digit where the Ids differed.
		\item the new ID is now only $\log\log n$ bits long. This is repeated until there are only six distinct numbers left. This takes $\log^*$ rounds each taking only constant time.
		\item each parent sends its number to its children which relabel themselves accordingly
		\item This is repeated another time and the IDs are taken $\mod 3$ resulting in a three colourin
	\end{enumerate}
	\begin{definition}
		The collection of the initial states of all nodes in the $r$-neighbourhood of a node $v$ is the $r$-hop view of $v$.
	\end{definition}
	\begin{definition}
		Let $\mathcal{G}$ be a family of network graphs. The $r$-neighbourhood graph $N_r(\mathcal{G})$ is defined as follows:\\
		The node set is the set of all possible labelled $r$-neighbourhoods (i.e. all possible $r$-hop views). There is an edge between tow labelled $r$-neighbourhoods $V_r$ and $V_r'$ if $V_r$ and $V_r'$ can be the $r$-hop views of adjacent nodes.
	\end{definition}
	\begin{lemma}
		For a given family of network graphs $\mathcal{G}$ there is an $r$-round algorithm that colours graphs of $\mathcal{G}$ with $c$ colours of the chromatic number of the neighbourhood graph is $\chi(N_r(\mathcal{G})) \leq c$.
	\end{lemma}
	\begin{definition}
		We define a directed graph $B_k$ which is closely related to the neighbourhood graph. The vertex set is made up of all $k$-tuples consisting increasing node labels. For two nodes $\alpha = (\alpha_1, ..., \alpha_k)$ and $\beta = (\beta_1, ..., \beta_k)$ there is an edge from $\alpha$ to $\beta$ if $\forall i$ it holds that $\beta_i = \alpha_{i+1}$.
	\end{definition}
	\begin{lemma}
		Viewed as an undirected graph, $B_{2r+1}$ is a subgraph of the $r$-neighbourhood graph of directed rings with $n$ nodes.
	\end{lemma}
	\begin{lemma}
		If $n>k$ the graph $B_{k+1}$ can be defined as the line graph $\mathcal{L}(B_k)$ of $B_k$.
	\end{lemma}
	\begin{lemma}
		It holds that \[\chi(\mathcal{L}(G)) \geq \log_2(\chi(G))\]
	\end{lemma}
	\begin{lemma}
		For all $n\geq 1$ it holds that $\chi(B_1) = n$. Further for $n\geq k \geq 2$ it holds that $\chi(B_k) \geq \log^{(k-1)} n$.
	\end{lemma}
	\begin{theorem}
		Every deterministic distributed algorithm to colour a directed ring with at most 3 colours needs at least $\log^*(\frac{n}{2}) - 1$ rounds.
	\end{theorem}
	\begin{corollary}
		Every deterministic distributed algorithm to compute a maximal independent set on a directed ring needs at least $\log^*(\frac{n}{2}) - \mathcal{O}(1)$ rounds.
	\end{corollary}
	\subsection{MIS}
	The following randomized algorithm gives a good solution to the maximum independent set.
	\begin{enumerate}
		\item the algorithm operates in synchronous rounds grouped into phases
		\item each node marks itself with probability $\frac{1}{2d(v)}$
		\item if no higher degree neighbour of $v$ is marked, node $v$ unmarks itself again
		\item delete all nodes that joined the MIS and their neighbours as the cannot join the MIS any more
	\end{enumerate}
	\begin{lemma}
		A node $v$ joins the MIS in step 3 with probability $p \geq \frac{1}{4d(v)}$
	\end{lemma}
	\begin{lemma}
		A node is called good if \[\sum_{w \in N(v)} \frac{1}{2d(v)} \geq \frac{1}{6}\]
		A good node will be removed in Step 4 with probability $p \geq \frac{1}{36}$.
	\end{lemma}
	\begin{lemma}
		An edge is called bad if both its endvertices are bad. Otherwise it's called good. At any time at least half of the edges are good.
	\end{lemma}
	\begin{lemma}
		A bad node has out-degree at least twice its in-degree.
	\end{lemma}
	\begin{lemma}
		The algorithm terminates in expectation in $\mathcal{O}(\log n)$ rounds.
	\end{lemma}
	\section{Consensus}
	In a distributed system with each node starting with input $x_i$, we speak of consensus if an algorithm can achieve the following properties 
	\begin{enumerate}
		\item Agreement: all alive nodes decide on a single value $x$
		\item Validity: the decided value $x$ is one of the initial inputs
		\item Termination: each vertex terminates at some point (either voting for one value or crashing)
	\end{enumerate}
	The following randomized consensus algorithm works in an asynchronous setting with less than half the nodes crashing
	\begin{enumerate}
		\item input bit $v_i \in \{0,1\}$, $round = 1$, decided = false
		\item broadcast $(v_i, round)$
		\item while true
		\item wait until majority of messages of current round arrived
		\item if all messages contain the same value $v$:  
		\item propose $(v,round)$, decided = true
		\item else: 
		\item propose $(\bot, round)$ //$\bot$ is a signal of disagreement
		\item end if 
		\item wait until a majority of proposals of current round arrived
		\item if all messages propose the same value $v$:
		\item $v_i = v$, decide = true
		\item else if there is at least one proposal for $v$:
		\item $v_i = v$
		\item else:
		\item choose $v_i$ uniformly at random
		\item end if
		\item $round = round +1$
		\item broadcast $(v_i,round)$
		\item end while
	\end{enumerate}
	\begin{theorem}
		The above algorithm satisfies validity, termination and comes to an agreement. In expectation it takes exponential time.
	\end{theorem}
	\subsection{shared coin}
	The following algorithm allows a dynamic network to use the same coin for all vertices at the same time. Here $f$ is the number of nodes that can turn byzantine. It should hold that $f \leq \frac{n}{3}$.
	\begin{enumerate}
		\item choose local coin $c_u = 0$ with probability $\frac{1}{n}$
		\item broadcast $c_u$
		\item wait for $n-f$ coins and store them in the local coin set $C_u$
		\item broadcast $C_u$
		\item wait for $n-f$ coin sets
		\item if at least one coin is 0 among all coins in $C_u$:
		\item return 0
		\item return 1
		\item end if
	\end{enumerate}
	\subsection{byzantine consensus}
	\begin{definition}
		A node which can have arbitrary or malicious behaviour is called \underline{byzantine}. This includes not sending messages, sending wrong messages, sending different messages to different neighbours and many more. A node that is not byzantine is called \underline{correct} or \underline{truthful}.
	\end{definition}
	The following probabilistic algorithm achieves consensus in an asynchronous setting with $<\frac{n}{9}$ byzantine nodes.
	\begin{enumerate}
		\item $x_i \in \{0,1\}$, $r = 1$, decided = false
		\item propose($x_i,r$)
		\item while not decided
		\item wait until $n-f$ proposals of current round $r$ arrived
		\item if at least $n-2f$ proposals contain the same value $x$: $x_i=x$ decided = true
		\item elseif at least $n-4f$ proposals contain the same value $x$: $x_i=x$
		\item else: choose $x_i$ randomly with $\mathbb{P}[x_i=0] = \mathbb{P}[x_i=1] = \frac{1}{2}$
		\item endif
		\item $r = r+1$, propose($x_i, r$)
		\item endwhile
		\item decision = $x_i$
	\end{enumerate}
	\begin{lemma}
		Let $f < \frac{n}{9}$. If a correct node chooses value $x$ in line 6, then no other correct node chooses value $y\neq x$ in line 6.
	\end{lemma}
	\begin{theorem}
		The algorithm solves binary agreement for up to $f<\frac{n}{9}$ byzantine nodes. 
	\end{theorem}
	\begin{definition}
		$N[u] = N(u) \cup \{u\}$
	\end{definition}
\section{Dominating Set}
	The following algorithm gives an approximation to a minimal dominating set. To this end, we colour vertices white in the beginning, black if they are added to the dominating set $S$ and grey if they are covered by a neighbour in $S$. For a vertex $u$ we define $W(u) = \{v \in N[u]\mid v \text{ is white}\}$.
	\begin{enumerate}
		\item while $v$ has white neighbours
		\item compute $\left|W(v)\right|$ and send it to all neighbours at distance at most 2
		\item if $\left|W(v)\right|$ is largest among neighbours of distance 2
		\item join $S$
		\item endif
		\item endwhile
	\end{enumerate}
	\begin{theorem}
		Let $S^*$ be the optimal dominating set and $S$ the one returned by the algorithm. Then $\frac{\left|S\right|}{\left|S^*\right|} \leq \ln \Delta + 2$. The algorithm takes $\Theta(n)$ rounds.
	\end{theorem}
	In the following we try to push this runtime to sublinear.
	\subsection{Fast Dominating Set Algorithm}
	\begin{enumerate}
		\item $W(v) = N[v]$, $w(v) = \left|W(v)\right|$
		\item while $W(v) \neq \varnothing$
		\item $w'(v) = w(v)$ rounded down to the nearest power of 2
		\item if $w(v) = \max_{u \in N_2(v)} w'(u)$ then $v.active = true$
		\item else $v.active = false$
		\item endif
		\item compute active neighbours $a(v) = \{u \in N(v) \mid u.active\}$
		\item $v.candidate = false$
		\item if $v.active = true$ then
		\item $v.candidate = true$ with probability $\frac{1}{\max_{u \in W(v)}a(u)}$
		\item endif
		\item compute $c(v) = \left|\{u \in W(v) \mid u.candidate\}\right|$
		\item if $v.candidate$ and $\sum_{u \in W(v)} c(u) \leq 3w(v)$ then
		\item node $v$ joins dominating set
		\item endif
		\item update $W$, $w$
		\item endwhile
	\end{enumerate}
	\begin{theorem}
		The algorithm computes a dominating set of size at most $(6\ln \Delta + 12)\left|S^*\right|$.
	\end{theorem}
	\begin{lemma}
		Consider an iteration of the while loop. Suppose that a node $u$ is white and that $2a(u) \geq \max_{v \in C(u)} \max_{y \in W(y)} a(y)$ where \[C(u) = \{v \in N(u) \mid v.candidate\}\] Then the probability that $u$ becomes dominated in this iteration is larger than $\frac{1}{9}$.
	\end{lemma}
	\section{Maximal matching}
	This section introduces a new technique called rounding. The idea is to solve a given integral problem as a continuous problem and then rounding the results to the nearest integer. This often allows for polylogarithmic time complexity.
	\begin{definition}
		A maximal matching is a subset $S$ of edges s.t. no vertex has two incident edges in $S$. Furthermore, there are no edges $e \in E\setminus S$ that can be added to $S$ without breaking the first condition. 
	\end{definition}
	This problem is a typical integer linear problem, i.e. one where variables $x_e$ for $e \in E$ are in $\{0,1\}$. The idea is now to allow continuous variables and fix the result by rounding. A non-integral result is called a fractional matching.
	\begin{definition}
		In a fractional matching we call a vertex $v$ loose if $c_v = \sum_{e \in E(v)} x_e \leq \frac{1}{2}$. An edge is called loose if both its vertices are loose. We call a fractional matching $f$-fraction if $c_v \geq f$ $\forall v \in V$.
	\end{definition}
	The following algorithm runs in $\mathcal{O}(\log n)$ time and yields a 4-approximation to for a fractional matching.\\
	\begin{enumerate}
		\item $x_e = 2^{-(\lceil\log \Delta \rceil)}$
		\item while both endpoints of $e$ are loose:
		\item $x_e = 2x_e$
		\item endwhile
	\end{enumerate}
	\begin{theorem}
		The algorithm computes a 4-approximation $\frac{1}{\Delta}$-fractional matching in $\mathcal{O}(\log \Delta)$ time.
	\end{theorem}
	The idea is now to get rid of the fractional edges. This works by starting to either multiply all edges of value $\frac{1}{\Delta}$ by a factor of 2 or by rounding them down to 0. In the next step this is done for edges of value $\frac{2}{\Delta}$ and so on.
	\begin{definition}
		We define the subgraph $G_f$ as the graph induced by all edges of value $f$ in $G$.
	\end{definition}
	\begin{definition}
		Rounding the graph $G_f$ means to identify a subset of edges of value $f$ in $E_f$ which will be doubled. All other edges in (which are also of value $f$!) will be assigned value 0. The resulting graph should still be a valid $2f$-fractional matching.
	\end{definition}
	\begin{definition}
		A perfect rounding of a graph $G$ is a rounding of the graph such that for all nodes $v$ half of its edges are assigned twice its value and the other half 0. Notice that $c_v$ and the size of the matching remain unchanged.
	\end{definition}
	We start by introducing the idea for bipartite graphs. For this, we need to construct a 2-decomposition of the graph $G_f$.
	\begin{definition}
		For a graph $G$ we define a decomposition $G'$ of $G$ by copying each vertex $v$ $\frac{d(v)}{2}$ times. The edges incident to $v$ are distributed among all these copies such that each copy has degree 2 (in the case that $d(v)$ is odd, one copy may have degree 1).
	\end{definition}
	Note that in $G'$ each vertex $v$ has $d(v) \in \{1,2\}$. I.e. $G'$ is a disjoint union of cycles and paths.
	\begin{definition}
		A cycle or path is called short if its length is at most $l = 24\log \Delta$ and long otherwise.
	\end{definition}
	Since we assumed that the starting graph is bipartite, all cycles are even. Therefore we would want the edge values to be raised and dropped alternately along the cycle. This is a perfect rounding. If the cycle is short, the following algorithm achieves this in $\mathcal{O}(\log \Delta)$.
	\begin{enumerate}
		\item Orient the cycle in one direction
		\item if $e$ goes from a node with colour 1 to colour 2:
		\item $x_e := 2x_e$
		\item else: $x_e := 0$
	\end{enumerate}
	In a long cycle the first line is not easy to solve. Therefore, we contend ourselves with a common direction only on subpaths of length at least $l$ on long cycles.
	\begin{definition}
		Consider a cycle with an orientation of each edge. A maximal directed path is a directed path that can not be extended since both neighbouring edges have inconsistent orientation.
	\end{definition}
	Starting from a random orientation we can achieve long directed paths by determining the length of a subpath and compute the length of the subpath it points towards. By flipping the edges of the shorter path, we create a longer directed path.
	\begin{enumerate}
		\item orient $e$ arbitrarily
		\item for $i=1,...,\log(l)$:
		\item compute the length of the path pointing in the opposite direction and flip the edges of the shorter path
		\item endfor
	\end{enumerate}
	Since we can now compute long subpaths of long cycles, we discuss rounding of long cycles in the next step.
	\begin{enumerate}
		\item compute long paths
		\item if $e$ is a boundary edge or goes from a node of colour 2 to colour 1: $x_e := 0$
		\item else: $x_e := 2x_e$
	\end{enumerate}
	\begin{lemma}
		Rounding long cycles leads to a loss of $\leq \frac{3}{l} \sum_{e \in E} x_e$.
	\end{lemma}
	Obviously the same approach works for long paths.\\
	The algorithm for short paths is a bit more complicated.
	\begin{enumerate}
		\item orient the graph with start node $s$ and end note $t$
		\item if $e$ is first edge:
		\item if $s$ is tight (\textit{not loose}): $x_e := 0$
		\item else: $x_e := 2x_e$
		\item else if $e$ is the last edge:
		\item if $t$ is tight: $x_e := 0$
		\item else: $x_e := 2x_e$
		\item else if $e$ is an even edge: $x_e := 0$
		\item else: $x_e := 2x_e$ 
	\end{enumerate}
	\begin{lemma}
		Rounding short paths results in a loss $\leq 4f \sum_{e \in E} x_e$
	\end{lemma}
	\begin{lemma}
		In the rounding step going from an $f$-fractional matching to a $2f$-fractional matching the matching decreases by a factor of at most $(1-\frac{3}{l} - 4f)$ and the rounding step takes $\mathcal{O}(\Delta)$ time.
	\end{lemma}
	\begin{lemma}
		This results in a $\frac{1}{16}$-fractional matching which is a constant factor smaller than the initial $\frac{1}{\Delta}$ matching. 
	\end{lemma}
	\begin{lemma}
		A constant factor approximation 16-fractional matching can be computed in $\mathcal{O}(\log^2 \Delta)$ in a 2-coloured bipartite graph.
	\end{lemma}
	\begin{lemma}
		A constant factor approximation matching can be computed in $\mathcal{O}(\log^2 \Delta)$ time in a 2-coloured bipartite graph with maximum degree $\Delta$.
	\end{lemma}
	What can we say about general graphs?\\
	The idea is to decompose the graph in a similar way to how it was done in the exercises. Assuming that the vertices are labelled, we can direct the edges from smaller to larger ID. Then each vertex is copied once. The first copy only keeps in-going edges, while the other copy only keeps out-going edges. It is clear that this construction generates a bipartite graph. This motivates the following definition.
	\begin{definition}
		The previous construction is called a bipartite double cover. The bipartition classes are obviously the set of vertices with in-edges or out-edges respectively. Since a vertex has only in-neighbours or only out-neighbours, it is easy to assign a 2-colouring.
	\end{definition}
	Since we can compute a $c$-approximation matching for bipartite graphs, the following lemma follows quickly.
	\begin{lemma}
		A 3c-approximation matching can be computed in $\mathcal{O}(\log^*n + \log^2\Delta)$ time in general graphs.
	\end{lemma}
	By repeatedly applying this lemma, we can even prove the following:
	\begin{theorem}
		A maximal matching can be computed in $\mathcal{O}((\log^*n + \log^2\Delta)\log n)$ time.
	\end{theorem}
	\section{Wireless networks}
	In a wireless network nodes can send messages only to nodes within a given range. Even if we assume that the transmission graph is complete, multiple nodes sending messages at the same time can lead to interference. Therefore we want that in each round all but one vertex only receive messages while the remaining one is free to communicate. How can we guarantee message transmissions? Communication is possible if a vertex of ID $j$ transmits a message in round $j$ but this needs a linear number of rounds. Trough randomization this can be improved substantially.
	\subsection{Wireless leader election}
	\begin{enumerate}
		\item In the first phase, the nodes transmit with probability $\frac{1}{2^{2^0}},\frac{1}{2^{2^1}},\frac{1}{2^{2^2}},...$ until no node transmits which yields a first approximation of the number of nodes
		\item in the second phase, a binary search is performed to determine an even better approximation of $n$. The first phase returned the search window $[l,u]$ in which the number of vertices must lie.
		\item in the third phase we use a biased random walk to give a final approximation of $n$. The second phase returned an approximation $d$ of the number of nodes. With probability $\frac{1}{2^d}$ each node now communicates. If nothing was transmitted, decrease $d$ and if more than one node communicated, increase $d$ until exactly one message was sent.
	\end{enumerate}
	Let $X$ denote the random variable representing the number of nodes transmitting in the same time slot.
	\begin{lemma}
		During phase 2, if the current approximation $j$ of $n$ is larger than $\log n + \log \log n$ or if during the first phase the current approximation $i$ is larger than $2\log n$ it holds that $\mathbb{P}[X>1] \leq \frac{1}{\log n}$. 
	\end{lemma}
	\begin{lemma}
		If $j<\log n - \log \log n$ or $i < \frac{1}{2} \log n$ then $\mathbb{P}[X=0] \geq \frac{1}{n}$
	\end{lemma}
	\begin{lemma}
		Let $v$ be such that $2v-1 \leq n \leq 2v$. If during the third phase the current approximation $d$ satisfies $d > v+2$ then $\mathbb{P}[X>1] \leq \frac{1}{4}$.
	\end{lemma}
	\begin{lemma}
		If $d<v-2$ then $\mathbb{P}[X=0] \leq \frac{1}{4}$
	\end{lemma}
	\begin{lemma}
		If $v-2 \leq d \leq v+2$ then $\mathbb{P}[X=1]$ is constant.
	\end{lemma}
	\begin{lemma}
		With probability $1-\frac{1}{\log n}$ we find a leader in phase 3 in $\mathcal{O}(\log \log n)$ time.
	\end{lemma}
	\begin{theorem}
		The algorithm elects a leader with probability of at least $1-\frac{\log \log n}{\log n}$ in $\mathcal{O}(\log \log n)$ time.
	\end{theorem}
	\subsection{Advice Algorithms}
	In an asynchronous deterministic setting with at least one crash it is not possible to solve consensus. In the following we will see that in a deterministic setting that allows for advice we can achieve different results.\\
	There are two possibilities for consensus. In the first case we have advice from the beginning on. That means, we always have some kind of trustworthy information. In case two, we deal with eventual advice. That means there will be a point in time at which there is secure information on which we can fall back on.\\
	The following chapter deals with an asynchronous message-passing system. At most $f$ processes may fail by halting. Each process has access to a failure detector. This is an unreliable oracle that gives the vertex continuous updates about the status of other processes. For each node the oracle can tell its process if it suspects the node to have crashed. If not, that node is considered trusted. A correct process never fails and a non-faulty one may fail but has not as of this point in time. A variant of this are limited-scope failure detectors. These can only give an answer to process that are ``reachable'' what ever that may mean.
	\begin{definition}[configuration]
		We say that a system is fully defined at any point in time by its configuration $C$. This includes the state of every node and all messages that are in transit.
	\end{definition}
	\begin{definition}[univalent]
		We call a configuration $C$ univalent if the decision value (think consensus) is determined independently of what happens later on. 
	\end{definition}
	\begin{remark}
		\begin{itemize}
			\item a configuration $C$ that is univalent with value $v$ is called $v$-valent
			\item $C$ can be univalent even though no node knows about it
		\end{itemize}
	\end{remark}
	\begin{definition}
		A configuration $C$ that is not univalent is called bivalent (assuming that the decision space is $\{0,1\}$).
	\end{definition}
	\begin{lemma}
		There is at least one selection of input vales such that the according initial configuration $C_0$ is bivalent if the number of crashes is $f\geq 1$.
	\end{lemma}
	\begin{definition}[transition]
		A transition from a configuration $C$ to a following configuration $C_\tau$ is characterized by an event $\tau = (u,m)$ meaning that node $u$ receives message $m$.
	\end{definition}
	\begin{remark}
		\begin{itemize}
			\item A transition $\tau = (u,m)$ is only applicable to configuration $C$ if $m$ was still in transit in $C$.
			\item The difference between $C$ and $C_\tau$ is that in $C_\tau$ $u$ might have a different state, $m$ is no longer in transit and there are potentially new messages in transit sent from $u$.
		\end{itemize}
	\end{remark}
	\begin{remark}
		\begin{enumerate}
			\item the set of configurations can be viewed as vertices of a graph and the transitions can be viewed as edges. The resulting graph is the so-called transition tree
			\item leaves are configurations where the execution terminates
		\end{enumerate}
	\end{remark}
	\begin{lemma}
		Assume two transitions $\tau_1,\tau_2$ for $u_1\neq u_2$ are both applicable to to $C$. Let $C_{\tau_1\tau_2}$ be the configuration that follows $C$ by first applying $\tau_1$ and then $\tau_2$ and define $C_{\tau_2\tau_1}$ analogously. Then $C_{\tau_1\tau_2} = C_{\tau_2\tau_1}$. 
	\end{lemma}
	\begin{definition}[critical configuration]
		We say that a configuration $C$ is critical, if $C$ is bivalent but all configurations that are direct children of $C$ in the configuration tree are univalent.
	\end{definition}
	\begin{lemma}
		In a system with bivalent configurations, it has to reach a critical point in finite time otherwise it does not reach consensus.
	\end{lemma}
	\begin{lemma}
		If a configuration tree contains a critical configuration, crashing a single node can create a bivalent leaf, i.e. a crash prevents the algorithm from reaching an agreement.
	\end{lemma}
	\begin{theorem}
		There is no deterministic algorithm which always achieves consensus in the asynchronous model with $f>0$.
	\end{theorem}
	Here, we are considering a system that clusters the vertices. Each cluster includes one correct process that is never suspected to have failed by any other process. However, that may only be the case after a time period $t$. A model with $q$ clusters can be understood as a network composed of $w$ disjoint LANs. 
	\begin{definition}[strong completeness]
		At some point every crashed participant will be suspected to be permanently dead.
	\end{definition}
	\begin{definition}[$k$-set agreement]
		In a network where every vertex starts with an initial value that is not necessarily binary, the vertices should agree on at most $k$ different values. For $k=1$ this is simply consensus.
	\end{definition}
	The solvability of $k$-set agreement now depends on $q,k$ and $x = \left|\bigcup_{j=1}^Q q_j\right|$ the total size of all clusters. This system satisfies the above strong completeness and one of the following two weak accuracies.
	\begin{definition}[perpetual weak $(x,q)$-accuracy]
		Some correct process in each cluster is never suspected by any process in that cluster.
	\end{definition}
	\begin{definition}[eventual weak $(x,y)$-accuracy]
		There is a time after which some correct process in each cluster is never suspected by any process in that cluster.
	\end{definition}
	We focus on two different classes of failure detection:
	\begin{itemize}
		\item $S_{x,q}$: strong completeness and perpetual weak $(x,q)$-accuracy
		\item $\diamond S_{x,q}$: strong completeness ad eventual weak $(x,q)$-accuracy
	\end{itemize}
	Under the first system, it is possible to solve $k$-set agreement with up to $f<k-1+x-q$ failures if $q<k$ and $f<x$ otherwise. Under the second system however, $f<\min\{\lceil\frac{n+1}{2}\rceil,k-1+x-q\}$ failures can be tolerated.
	\subsection{$k$-set-agreement with advice}
	\begin{definition}
		Suppose each vertex $i$ starts with a value $x_i$. A solution to the $k$-set agreement problem satisfies
		\begin{itemize}
			\item \underline{agreement}: all vertices agree on at most $k$ many distinct values
			\item \underline{all-same-validity}: the $k$ values must be a subset of the initial values
			\item \underline{termination}: every correct node eventually decides
		\end{itemize}
	\end{definition}
	The following algorithm works for th $S_{x,q}$ failure detector.
	\begin{enumerate}
		\item $r= 1; s= 0; e=x_i; if q \leq k, b= \min\{n, k - 1 + x - q\} else b= \min\{n, x - 1\}$
		\item for each set $S$ of size $b$ do
		\item if $i \in S$
		\item for each $X\subset S$ of size $\min\{k,q\}$ do
		\item if $i \in X$ then $\forall p \in S$ broadcast $(e,r,s)$
		\item else $e=e'$ from $(r,s)$-broadcast where $e'$ is from $p \in X$ not suspected
		\item $s=s+1$
		\item endfor
		\item broadcast $(e,r)$
		\item else $e=e'$ from some other broadcast
		\item $r=r+1$
		\item endfor
	\end{enumerate}
	\begin{theorem}
		This algorithm solves $k$-set agreement in an asynchronous setting with a failure tolerance of up to $f<\min \{n,k-1+x-q\}$ if $q\leq k$ and $f<\min\{n,x-1\}$ otherwise.
	\end{theorem}
	The following algorithm can solve $k$-set-agreement with $\diamond S_{x,q}$ fault detection.
	\begin{enumerate}
		\item $r=1$, $e=x_i$
		\item while TRUE do
		\item $h=\varnothing$
		\item $e=S_{x,q}(id,e)$
		\item for all permutations $t$ of $[n]$ do
		\item broadcast $(r,t,i,e,h)$, $r=r+1$
		\item define $M,S,H$ as the sets of messages, participants and histories of the first $\lceil\frac{n+1}{2}\rceil$ messages received
		\item $e=$ message received by the participant in the first permutation $t$ of $[n]$
		\item $E=$ estimates of $S$ from last round and last permutation in $H$
		\item if $0<\left|E\right| \leq k$ broadcast $e$, return $e$
		\item if $e'$ received, then $e=e'$ and broadcast $e$, return $e$
		\item endfor
		\item endwhile
	\end{enumerate}
	\begin{theorem}
		This algorithm solves $k$-set-agreement asynchronously while tolerating up to $fy\min\{\frac{n}{2},k-1+x-q\}$ if $q\leq k$ and $f<\min\{\frac{n}{2}, x-1\}$ otherwise.
	\end{theorem}
	\subsection{combinatoric topology}
	\begin{definition}
		An $n$-simplex $S$ is a space spanned by a collection of $n+1$ affine independent points $\{v_1,...,v_{n+1}\}$. That is \[S = \left\{\sum_{i=1}^{n+1} c_i v_i \mid \sum_{i=1}^{n+1} c_i = 1, \; c_i \geq 0 \; \forall i\right\}\]
		A simplicial subdivision of $S$ is a partition of $S$ into smaller sub-simplices such that two of them are either disjoint or share a full face. Finally, a Sperner-colouring of a simplicial subdivision of $S$ is one that assigns different colours to all of the initial $n+1$ vertices of $S$ and assigns each subdivision vertex of a face $F$ of $S$ one of the colours of the boundary vertices of $F$. There are no restrictions on the colours of internal vertices.
	\end{definition}
	\begin{lemma}
		A simplicial complex of an $n$-simplex $S$ as in the previous definition together with a Sperner-colouring always admits a simplicial cell with $n+1$ differently coloured vertices.
	\end{lemma}
	\section{Sorting and Counting Networks}
	\begin{definition}
		A comparator is a device with two inputs, $x$ and $y$ and two outputs $x',y'$ such that $x'=\min\{x,y\}$ and $y'=\max\{x,y\}$. We construct so-called comparison networks that consist of wires that connect comparators.
	\end{definition}
	\begin{definition}
		The depth of an input wire is 0. Teh depth of a comparator is the maximum depth of its input wires plus one. The depth of an output wire is the depth of the previous comparator. The depth of a comparison network is the maximum depth of the output wires.
	\end{definition}
	\begin{definition}
		A bitonic sequence s a sequence of numbers that first monotonically increases and then monotonically decreases or vice versa.
	\end{definition}
	\begin{definition}[half-cleaner]
		A half-cleaner is a comparison network of depth 1 that compares wire $i$ with wire $i+\frac{n}{2}$.
	\end{definition}
	In the following, we will consider binary bitonic sequences, that is of the form $0^j1^k0^l$ or $1^j0^k1^l$.
	\begin{lemma}
		Feeding a bitonic sequence into a half-cleaner will clean either the upper or the lower half of the $n$-wires. That is, it makes half the sequence all 0 (or all 1 respectively). The other half is bitonic.
	\end{lemma}
	\begin{definition}[Bitonic sequence sorter]
		A bitonic sequence sorter of width $n (=2^k)$ consists of a half cleaner of width $n$ and then two bitonic sequence sorters of width $\frac{n}{2}$ each.
	\end{definition}
	\begin{lemma}
		A bitonic sequence sorter of width $n$ has depth $\log n$.
	\end{lemma}
	In order to sort arbitrary sequences, we now need to introduce another concept, called merging networks.
	\begin{definition}[merger]
		A merger is a network of depth 1 that compares wire $i$ with wire $n-i+1$.
	\end{definition}
	\begin{definition}
		A merging network is a merger of depth 1 followed by two bitonic sequence sorters of depth $\frac{n}{2}$ each. That is, it is a bitonic sequence sorter where we replace the first half-cleaner by a merger.
	\end{definition}
	\begin{lemma}
		A merging network of depth $n$ merges two sorted input sequences of length $\frac{n}{2}$ each into one sorted sequence of length $n$.
	\end{lemma}
	It should now be obvious that applying the previous lemma recursively allows us to fully sort any sequence.
	\begin{definition}[Batcher's network]
		A Batcher's sorting network of width $n$ consists of two Batcher's sorting networks of width $\frac{n}{2}$ followed by a merger. A network of width 1 is empty. 
	\end{definition}
	\begin{lemma}
		In order to sort a sequence of length $n$ we need a sorting network of depth $\mathcal{O}(\log^2 n)$.
	\end{lemma}
	\section{Minimum Spanning Tree}
	The idea is to use an approach similar to Kruskal. We start with $n$ components. In each round we can merge components by picking the smallest edge leaving a component. This however, may create very large components along the way. Therefore, it is smart to restrict the size of components to $\sqrt{n}$. This ensures that communication is sublinear in each round. 
	\begin{definition}
		Consider a graph $G$ and a partition of $V$ into disjoint subsets $S_1,...,S_N$ inducing connected subgraphs $G[S_1],...,G[S_n]$.. We define an $\alpha$-congestion shortcut with dilation $\beta$ to be a set of subgraphs $H_1,...,H_N$ one for each set $S_i$ such that \begin{enumerate}
			\item for each $i$ the diameter of the subgraph $G[S_i]+H_i$ is at most $\beta$
			\item for each $e \in E$ the number of subgraphs $G[S_i]+H_i$ containing $e$ is at most $\alpha$.
		\end{enumerate}
	\end{definition}
	\begin{theorem}
		Suppose that the graph family $\mathcal{G}$ is such that for each graph $G \in \mathcal{G}$ and any partition of $G$ into vertex-disjoint connected subsets $S_1,...,S_N$ we can find an $\alpha$-congestion $\beta$-dilation shortcut such that $\max\{\alpha,\beta\}\leq K$. Here, $K$ can be a function of the family $\mathcal{G}$. 
	\end{theorem}
	\section{Dynamic networks}
	\begin{definition}
		A dynamic graph $G$ is called $T$-interval connected for $T \in \mathbb{N}$ if for all $r\in \mathbb{N}$ the static graph $G_{r,T} = \left(V, \bigcap_{i=r}^{r+T-1} E(i)\right)$ is connected. If $G$ is 1-interval connected we say that $G$ is always connected.
	\end{definition}
	\begin{theorem}
		Assume that there is a single token in the network. Further assume that at time 0 at least one node knows the token and that once they know the token, all nodes broadcast it in every round. In a 1-interval connected graph $G=(V,E)$ with $n$ nodes, after $r\leq n-1$ rounds, at least $r+1$ nodes know the token. Hence, in particular after $n-1$ rounds, all nodes know the token.
	\end{theorem}
	Now, suppose that the graph is $T$-interval connected.
	\begin{enumerate}
		\item $S = \varnothing$
		\item for $i=0,...,\lceil\frac{k}{T}\rceil-1$ do
		\item for $r=0,...,2T-1$ do
		\item if $S \neq A$
		\item $t = \min\{A \setminus S\}$
		\item broadcast $t$
		\item $S = S \cup \{t\}$
		\item end if
		\item receive $t_1,...,t_s$ from neighbours
		\item $A = A \cup \{t_1,...,t_s\}$
		\item end for
		\item $S = \varnothing$
		\item end for
	\end{enumerate}
\end{document}