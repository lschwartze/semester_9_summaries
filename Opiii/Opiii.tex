\documentclass[a4paper, 12pt]{article}

\usepackage{fullpage}
\usepackage[utf8]{inputenc}
\usepackage[english]{babel}
\usepackage{amsmath,amssymb}
\usepackage[explicit]{titlesec}
\usepackage{ulem}
\usepackage[onehalfspacing]{setspace}
\usepackage{amsthm}

\theoremstyle{plain}
\newtheorem{theorem}{Theorem}[section] % reset theorem numbering for each chapter

\theoremstyle{definition}
\newtheorem{definition}[theorem]{Definition} % definition numbers are dependent on theorem numbers
\theoremstyle{lemma}
\newtheorem{lemma}[theorem]{Lemma}

\theoremstyle{remark}
\newtheorem{remark}[theorem]{Remark}

\theoremstyle{corollary}
\newtheorem{corollary}[theorem]{Corollary}

\theoremstyle{example}
\newtheorem{example}[theorem]{Example}

\titleformat{\subsection}
{\small}{\thesubsection}{1em}{\uline{#1}}
\begin{document}
	\begin{titlepage} 
		\title{Opiii}
		\clearpage\maketitle
		\thispagestyle{empty}
	\end{titlepage}
	\tableofcontents
	\newpage
	\section{Dynamic Networks}
	Dynamic graph networks are graph networks that change over time. Communication is in synchronous, asynchronous or semi-synchronous rounds. Additionally shared memory is possible. Network elements may be failure-free or failure-prone. A classical example are \underline{mobile ad-hoc networks}. Those are temporary interconnection networks of mobile wireless nodes without a fixed infrastructure. Communication happens whenever mobile nodes come within the wireless range of each other.
	\begin{example}
		In mobile ad hoc networks, one may want to colour the graph or maintain a routing mechanism for communication to any particular destination in the network.
	\end{example}
	\subsection{Almost constant message-passing vertex colouring in a tree}
	Let $T$ be a tree network with $n$ labelled vertices in $[n]$. Colouring the graph can be done in almost constant, i.e. in $\log^*$ time.
	\begin{definition}
		$\log^*(x)$ is defined as the number of $\log$ functions that need to be applied to $x$ such that the result is at most 1. E.g. $\log^*(16) = 3$ and $\log^*2^{65536} = 5$.
	\end{definition}
	\begin{enumerate}
		\item begin by rooting the tree at vertex $0$. This defines an order on the tree
		\item each parent sends its number to all of its children
		\item each child computes the smallest index $i$ where its number differs from the parent's number. It is important to note that this can be done in constant time with suitable hardware
		\item It computes a new ID for itself consisting of a trailing bit corresponding to the bit where IDs disagreed. The new ID begins with the binary representation of the digit where the Ids differed.
		\item the new ID is now only $\log\log n$ bits long. This is repeated until there are only six distinct numbers left. This takes $\log^*$ rounds each taking only constant time.
		\item each parent sends its number to its children which relabel themselves accordingly
		\item This is repeated another time and the IDs are taken $\mod 3$ resulting in a three colourin
	\end{enumerate}
	\begin{definition}
		The collection of the initial states of all nodes in the $r$-neighbourhood of a node $v$ is the $r$-hop view of $v$.
	\end{definition}
	\begin{definition}
		Let $\mathcal{G}$ be a family of network graphs. The $r$-neighbourhood graph $N_r(\mathcal{G})$ is defined as follows:\\
		The node set is the set of all possible labelled $r$-neighbourhoods (i.e. all possible $r$-hop views). There is an edge between tow labelled $r$-neighbourhoods $V_r$ and $V_r'$ if $V_r$ and $V_r'$ can be the $r$-hop views of adjacent nodes.
	\end{definition}
	\begin{lemma}
		For a given family of network graphs $\mathcal{G}$ there is an $r$-round algorithm that colours graphs of $\mathcal{G}$ with $c$ colours of the chromatic number of the neighbourhood graph is $\chi(N_r(\mathcal{G})) \leq c$.
	\end{lemma}
	\begin{definition}
		We define a directed graph $B_k$ which is closely related to the neighbourhood graph. The vertex set is made up of all $k$-tuples consisting increasing node labels. For two nodes $\alpha = (\alpha_1, ..., \alpha_k)$ and $\beta = (\beta_1, ..., \beta_k)$ there is an edge from $\alpha$ to $\beta$ if $\forall i$ it holds that $\beta_i = \alpha_{i+1}$.
	\end{definition}
	\begin{lemma}
		Viewed as an undirected graph, $B_{2r+1}$ is a subgraph of the $r$-neighbourhood graph of directed rings with $n$ nodes.
	\end{lemma}
	\begin{lemma}
		If $n>k$ the graph $B_{k+1}$ can be defined as the line graph $\mathcal{L}(B_k)$ of $B_k$.
	\end{lemma}
	\begin{lemma}
		It holds that \[\chi(\mathcal{L}(G)) \geq \log_2(\chi(G))\]
	\end{lemma}
	\begin{lemma}
		For all $n\geq 1$ it holds that $\chi(B_1) = n$. Further for $n\geq k \geq 2$ it holds that $\chi(B_k) \geq \log^{(k-1)} n$.
	\end{lemma}
	\begin{theorem}
		Every deterministic distributed algorithm to colour a directed ring with at most 3 colours needs at least $\log^*(\frac{n}{2}) - 1$ rounds.
	\end{theorem}
	\begin{corollary}
		Every deterministic distributed algorithm to compute a maximal independent set on a directed ring needs at least $\log^*(\frac{n}{2}) - \mathcal{O}(1)$ rounds.
	\end{corollary}
	\subsection{MIS}
	The following randomized algorithm gives a good solution to the maximum independent set.
	\begin{enumerate}
		\item the algorithm operates in synchronous rounds grouped into phases
		\item each node marks itself with probability $\frac{1}{2d(v)}$
		\item if no higher degree neighbour of $v$ is marked, node $v$ unmarks itself again
		\item delete all nodes that joined the MIS and their neighbours as the cannot join the MIS any more
	\end{enumerate}
	\begin{lemma}
		A node $v$ joins the MIS in step 3 with probability $p \geq \frac{1}{4d(v)}$
	\end{lemma}
	\begin{lemma}
		A node is called good if \[\sum_{w \in N(v)} \frac{1}{2d(v)} \geq \frac{1}{6}\]
		A good node will be removed in Step 4 with probability $p \geq \frac{1}{36}$.
	\end{lemma}
	\begin{lemma}
		An edge is called bad if both its endvertices are bad. Otherwise it's called good. At any time at least half of the edges are good.
	\end{lemma}
	\begin{lemma}
		A bad node has out-degree at least twice its in-degree.
	\end{lemma}
	\begin{lemma}
		The algorithm terminates in expectation in $\mathcal{O}(\log n)$ rounds.
	\end{lemma}
	\section{Consensus}
	In a distributed system with each node starting with input $x_i$, we speak of consensus if an algorithm can achieve the following properties 
	\begin{enumerate}
		\item Agreement: all alive nodes decide on a single value $x$
		\item Validity: the decided value $x$ is one of the initial inputs
		\item Termination: each vertex terminates at some point (either voting for one value or crashing)
	\end{enumerate}
	The following randomized consensus algorithm works in an asynchronous setting with less than half the nodes crashing
	\begin{enumerate}
		\item input bit $v_i \in \{0,1\}$, $round = 1$, decided = false
		\item broadcast $(v_i, round)$
		\item while true
		\item wait until majority of messages of current round arrived
		\item if all messages contain the same value $v$:  
		\item propose $(v,round)$, decided = true
		\item else: 
		\item propose $(\bot, round)$ //$\bot$ is a signal of disagreement
		\item end if 
		\item wait until a majority of proposals of current round arrived
		\item if all messages propose the same value $v$:
		\item $v_i = v$, decide = true
		\item else if there is at least one proposal for $v$:
		\item $v_i = v$
		\item else:
		\item choose $v_i$ uniformly at random
		\item end if
		\item $round = round +1$
		\item broadcast $(v_i,round)$
		\item end while
	\end{enumerate}
	\begin{theorem}
		The above algorithm satisfies validity, termination and comes to an agreement. In expectation it takes exponential time.
	\end{theorem}
	\subsection{shared coin}
	The following algorithm allows a dynamic network to use the same coin for all vertices at the same time. Here $f$ is the number of nodes that can turn byzantine. It should hold that $f \leq \frac{n}{3}$.
	\begin{enumerate}
		\item choose local coin $c_u = 0$ with probability $\frac{1}{n}$
		\item broadcast $c_u$
		\item wait for $n-f$ coins and store them in the local coin set $C_u$
		\item broadcast $C_u$
		\item wait for $n-f$ coin sets
		\item if at least one coin is 0 among all coins in $C_u$:
		\item return 0
		\item return 1
		\item end if
	\end{enumerate}
\end{document}