\documentclass[a4paper, 12pt]{article}

\usepackage{fullpage}
\usepackage[utf8]{inputenc}
\usepackage[english]{babel}
\usepackage{amsmath,amssymb}
\usepackage[explicit]{titlesec}
\usepackage{ulem}
\usepackage[onehalfspacing]{setspace}
\usepackage{amsthm}

\theoremstyle{plain}
\newtheorem{theorem}{Theorem}[section] % reset theorem numbering for each chapter

\theoremstyle{definition}
\newtheorem{definition}[theorem]{Definition} % definition numbers are dependent on theorem numbers
\theoremstyle{lemma}
\newtheorem{lemma}[theorem]{Lemma}

\theoremstyle{remark}
\newtheorem{remark}[theorem]{Remark}

\theoremstyle{corollary}
\newtheorem{corollary}[theorem]{Corollary}

\theoremstyle{example}
\newtheorem{example}[theorem]{Example}

\titleformat{\subsection}
{\small}{\thesubsection}{1em}{\uline{#1}}
\begin{document}
	\begin{titlepage} 
		\title{Opiii}
		\clearpage\maketitle
		\thispagestyle{empty}
	\end{titlepage}
	\tableofcontents
	\newpage
	\section{Dynamic Networks}
	Dynamic graph networks are graph networks that change over time. Communication is in synchronous, asynchronous or semi-synchronous rounds. Additionally shared memory is possible. Network elements may be failure-free or failure-prone. A classical example are \underline{mobile ad-hoc networks}. Those are temporary interconnection networks of mobile wireless nodes without a fixed infrastructure. Communication happens whenever mobile nodes come within the wireless range of each other.
	\begin{example}
		In mobile ad hoc networks, one may want to colour the graph or maintain a routing mechanism for communication to any particular destination in the network.
	\end{example}
	\subsection{Almost constant message-passing vertex colouring in a tree}
	Let $T$ be a tree network with $n$ labelled vertices in $[n]$. Colouring the graph can be done in almost constant, i.e. in $\log^*$ time.
	\begin{definition}
		$\log^*(x)$ is defined as the number of $\log$ functions that need to be applied to $x$ such that the result is at most 1. E.g. $\log^*(16) = 3$ and $\log^*2^{65536} = 5$.
	\end{definition}
	\begin{enumerate}
		\item begin by rooting the tree at vertex $0$. This defines an order on the tree
		\item each parent sends its number to all of its children
		\item each child computes the smallest index $i$ where its number differs from the parent's number. It is important to note that this can be done in constant time with suitable hardware
		\item It computes a new ID for itself consisting of a trailing bit corresponding to the bit where IDs disagreed. The new ID begins with the binary representation of the digit where the Ids differed.
		\item the new ID is now only $\log\log n$ bits long. This is repeated until there are only six distinct numbers left. This takes $\log^*$ rounds each taking only constant time.
		\item each parent sends its number to its children which relabel themselves accordingly
		\item This is repeated another time and the IDs are taken $\mod 3$ resulting in a three colourin
	\end{enumerate}
	\begin{definition}
		The collection of the initial states of all nodes in the $r$-neighbourhood of a node $v$ is the $r$-hop view of $v$.
	\end{definition}
	\begin{definition}
		Let $\mathcal{G}$ be a family of network graphs. The $r$-neighbourhood graph $N_r(\mathcal{G})$ is defined as follows:\\
		The node set is the set of all possible labelled $r$-neighbourhoods (i.e. all possible $r$-hop views). There is an edge between tow labelled $r$-neighbourhoods $V_r$ and $V_r'$ if $V_r$ and $V_r'$ can be the $r$-hop views of adjacent nodes.
	\end{definition}
	\begin{lemma}
		For a given family of network graphs $\mathcal{G}$ there is an $r$-round algorithm that colours graphs of $\mathcal{G}$ with $c$ colours of the chromatic number of the neighbourhood graph is $\chi(N_r(\mathcal{G})) \leq c$.
	\end{lemma}
	\begin{definition}
		We define a directed graph $B_k$ which is closely related to the neighbourhood graph. The vertex set is made up of all $k$-tuples consisting increasing node labels. For two nodes $\alpha = (\alpha_1, ..., \alpha_k)$ and $\beta = (\beta_1, ..., \beta_k)$ there is an edge from $\alpha$ to $\beta$ if $\forall i$ it holds that $\beta_i = \alpha_{i+1}$.
	\end{definition}
	\begin{lemma}
		Viewed as an undirected graph, $B_{2r+1}$ is a subgraph of the $r$-neighbourhood graph of directed rings with $n$ nodes.
	\end{lemma}
	\begin{lemma}
		If $n>k$ the graph $B_{k+1}$ can be defined as the line graph $\mathcal{L}(B_k)$ of $B_k$.
	\end{lemma}
	\begin{lemma}
		It holds that \[\chi(\mathcal{L}(G)) \geq \log_2(\chi(G))\]
	\end{lemma}
	\begin{lemma}
		For all $n\geq 1$ it holds that $\chi(B_1) = n$. Further for $n\geq k \geq 2$ it holds that $\chi(B_k) \geq \log^{(k-1)} n$.
	\end{lemma}
	\begin{theorem}
		Every deterministic distributed algorithm to colour a directed ring with at most 3 colours needs at least $\log^*(\frac{n}{2}) - 1$ rounds.
	\end{theorem}
	\begin{corollary}
		Every deterministic distributed algorithm to compute a maximal independent set on a directed ring needs at least $\log^*(\frac{n}{2}) - \mathcal{O}(1)$ rounds.
	\end{corollary}
\end{document}