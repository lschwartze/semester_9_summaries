\documentclass[a4paper, 12pt]{article}

\usepackage{fullpage}
\usepackage[utf8]{inputenc}
\usepackage[english]{babel}
\usepackage{amsmath,amssymb}
\usepackage[explicit]{titlesec}
\usepackage{ulem}
\usepackage[onehalfspacing]{setspace}
\usepackage{amsthm}

\theoremstyle{plain}
\newtheorem{theorem}{Theorem}[section] % reset theorem numbering for each chapter

\theoremstyle{definition}
\newtheorem{definition}[theorem]{Definition} % definition numbers are dependent on theorem numbers
\theoremstyle{lemma}
\newtheorem{lemma}[theorem]{Lemma}

\theoremstyle{remark}
\newtheorem{remark}[theorem]{Remark}

\theoremstyle{corollary}
\newtheorem{corollary}[theorem]{Corollary}

\theoremstyle{example}
\newtheorem{example}[theorem]{Example}

\titleformat{\subsection}
{\small}{\thesubsection}{1em}{\uline{#1}}
\begin{document}
	\begin{titlepage} 
		\title{Calculus of Variations}
		\clearpage\maketitle
		\thispagestyle{empty}
	\end{titlepage}
	\tableofcontents
	\newpage
	\section{Introduction}
	The objects of calculus of variations are so-called functionals, i.e. functions of functions. The main interests are the critical points of these functionals. Since such critical points are often solutions to (elliptic) PDEs. We will therefore look for minima and saddle points.
	\begin{example}[soap bubbles]
		Which object encloses a fixed volume and has smallest surface area?\\
		Formulation: \[\mathcal{F}: \{\text{surfaces enclosing volume } v_0\} \to \mathbb{R}_{\geq 0}\]
		The minimizer is a sphere.
	\end{example}
	\subsection{mathematical formulation}
	Let $X$ be a Banach space. Let $\varnothing \neq U \subset X$ and \[E: U \to \mathbb{R}\] be a functional.\\
	Now, if $E$ is bounded from below (i.e. $\exists M>0: E(u) \geq -M \; \forall u \in U$), is there a $u \in U$ s.t. $E(u) = \inf_{v \in U} E(v)$? Are there other critical points?\\
	There are two ways to approach the first question.
	\begin{enumerate}
		\item if $U$ is open and $E$ is differentiable then look for a solution to \[E'(u) = 0\] (called the classical method in CV). This will lead us to $u$ being a solution to a PDE.
		\item take a sequence $(u_n)_{n \in \mathbb{N}} \subset U$ s.t. $E(u_n) \to \inf_{v \in U} E(v)$ (a minimizing sequence).
		\begin{enumerate}
			\item Find a topology s.t. $\exists (u_{n_k})_{k \in \mathbb{N}}, u \in U$ with \[u_{n_k} \to u\] in this topology.
			\item check if $E(u) \leq \liminf_{k \to \infty} E(u_{n_k})$.
		\end{enumerate}
	\end{enumerate}
	\section{First variation and convexity}
	\subsection{First variation}
	We fix $X$ to be a Banach space
	\begin{definition}
		Let $\varnothing \neq V\subset X$ be open. Let $E: V \to \mathbb{R}$. We say that $E$ is \underline{Fréchet differentiable} at $u \in V$ if $\exists A \in L(X,\mathbb{R})$ (a linear bounded map) s.t. \[\lim_{||\varphi||_x \to 0} \frac{E(u + \varphi) - E(u) - A\varphi}{||\varphi||_x} = 0\]
	\end{definition}
	$A$ is called the Fréchet derivative of $E$ at $u$ denoted by $E'(u)$.
	\begin{definition}[First variation]
		Let $\varnothing \neq V \subset X$, $E:V \to \mathbb{R}$ and $u \in V$. Let $\varphi \in X$ s.t. \[\exists \delta > 0 \text{ s.t. } u + t\varphi \in V \; \forall t \in (-\delta, \delta)\]
		If \[(-\delta, \delta) \ni t \mapsto E(u+t\varphi) \in \mathbb{R}\] is differentiable at $t=0$ we say that $E$ has first in variation $u$ in direction $\varphi$ and write \[\delta E(u)(\varphi) = \frac{d}{dt} E(u+t\varphi)|_{t = 0}\]
	\end{definition}
	\begin{theorem}[Fundamental Lemma of CV]
		Let $\Omega \subset \mathbb{R}^n$ open, $w \in L_{loc}^1(\Omega)$ (i.e. $\forall K \subset \Omega$ compact, $w \in L^1(K))$. If \[\int_\Omega w\varphi \,dx = 0 \; \forall \varphi \in C_0^\infty(\Omega)\] then $w = 0$ a.e.
	\end{theorem}
	\begin{corollary}
		Let $n \in \mathbb{N}$, $u \in L^1_{loc}((a,b))$ s.t. \[\int_a^b u \frac{d^n}{dx^n} \varphi \,dx = 0 \,\forall \varphi \in c_0^\infty ((a,b))\] Then $\exists a_1,...,a_n \in \mathbb{R}$ s.t. \[u(x) = \sum_{i=0}^{n-1} a_i x^i\]
	\end{corollary}
	\begin{definition}
		Let $u \in L^1_{loc}(\Omega)$. We say that $u$ is once weakly differentiable if $\forall i \in \{1,2,...,n\}$ $\exists v_i \in L^1_{loc}(\Omega)$ s.t. \[\int_\Omega u \partial_{x_i}\varphi \, dx = - \int_\Omega v_i \varphi \,dx \; \forall \varphi \in C_0^\infty (\Omega)\] In that case $\partial_i u = v_i$.\\
		Similarly, if $\alpha \in \mathbb{N}_0^n$ we say $u$ is $\alpha$-weakly differentiable if \[\exists v_\alpha \in L^1_{loc} \text{ s.t. } \int_\Omega uD^\alpha \varphi \, dx = (-1)^{\left|\alpha\right|} \int v_\alpha \varphi \, dx\]
	\end{definition}
	\begin{definition}[Sobolev spaces]
		For $m \in \mathbb{N}$, $p \in (1,\infty)$, define \[W^{m,p}(\Omega) = \{u \in L^p(\Omega) \text{ s.t. } D^\alpha u \in L^p(\Omega) \; \forall \alpha \in \mathbb{N}_0^n, \; \left|\alpha\right| \leq m\}\] This is a vector space and can be equipped with the norm \[||u||_{W^{m,p}} = \left(\sum_{\left(\alpha\right) \leq m} ||D^\alpha u||^p_{L^p(\Omega)}\right)^{\frac{1}{p}}\]
	\end{definition}
	\begin{theorem}
		$(W^{m,p}(\Omega), ||\cdot||_{W^{m,p}})$ is a Banach space and for $p=2$ it's a Hilbert space. Finally, $C^\infty(\Omega) \cap W^{m,p}(\Omega)$ is dense in $W^{m,p}(\Omega)$.
	\end{theorem}
	\begin{definition}
		$W_0^{m,p}(\Omega) = \overline{C_0^\infty(\Omega)}^{W^{m,p}(\Omega)}$, i.w. $f \in W_0^{m,p}(\Omega)$ if \[\exists (f_n)_{n \in \mathbb{N}} \subset C_0^\infty (\Omega)\] s.t. \[||f-f_n||_{W^{m,p}} \to 0, \, n \to \infty\]
	\end{definition}
	\begin{remark}
		$W_0^{m,p}(\Omega) \subset W^{m,p}(\Omega)$
	\end{remark}
	\begin{definition}[weak solution of Poisson equation]
		We say that $u$ is a \underline{weak solution} of \[\begin{cases}
			-\nabla u &= f, \text{ in } \Omega\\
			u &= 0, \text{ on } \partial \Omega
		\end{cases}\] if $u \in W_0^{1,2}(\Omega)$ and \[\int_\Omega (\nabla u \nabla \varphi - f\varphi) \, dx = 0 \; \forall \varphi \in C_0^\infty(\Omega)\]
	\end{definition}
	\section{Direct Methods}
	\begin{lemma}[Poincaré-inequality]
		$\exists C = C(\Omega)$ s.t. \[||u||_{L^2} \leq C||\nabla u||_{L^2} \; \forall u \in W_0^{1,2}(\Omega)\]
	\end{lemma}
	\begin{lemma}
		Define $||u||_{W_0^{1,2}(\Omega)} = ||\nabla u||_{L^2}$, then $||\cdot ||_{W^{1,2}}$ and $||\cdot ||_{W_0^{1,2}}$ are equivalent in $W_0^{1,2}(\Omega)$.
	\end{lemma}
	\begin{theorem}
		Every bounded sequence in a Hilbert space admits a weakly convergent subsequence. That means given $(v_k)_{k \in \mathbb{N}} \subset H$ bounded in Hilbert space $H$ $\exists (v_{k_l})_{l \in \mathbb{N}} \exists v \in H$ s.t. \[\langle v_{k_l}, w\rangle_H \to \langle v, w\rangle_H \; \forall w \in H\]
	\end{theorem}
	\begin{theorem}
		If $(u_k)_{k \in \mathbb{N}}$ converges weakly to $u$ in $H$, then \[||u||_H \leq \liminf_{k \to \infty} ||u_k||_H\]
	\end{theorem}
	\begin{remark}
		A functional $T:X\to Y$ is compact if $\forall (x_k)_{k \in \mathbb{N}}$ bounded in $X$, the sequence $(Tx_n)_{n \in \mathbb{N}} \subset Y$ admits a convergent subsequence.
	\end{remark}
	\begin{theorem}
		The inclusion map $W_0^{1,2} \rightarrow L^2(\Omega)$ that maps $u \mapsto u$ is compact.
	\end{theorem}
	\subsection{Some concepts from functional analysis}
	Let $X$ be a $\mathbb{R}$-Banach space.\\
	\begin{definition}
		We define the dual space \[X' = L(X,\mathbb{R}) = \{T: X \to \mathbb{R} \text{ linear and continuous}\}\]
	\end{definition}
	\begin{lemma}
		For linear maps, the following are equivalent:
		\begin{enumerate}
			\item continuity
			\item continuity at 0
			\item boundedness
		\end{enumerate}
	\end{lemma}
	\begin{definition}
		Similarly, the bidual space \[X'' = L(X', \mathbb{R})\] is also a Banach space. There exists a canonical map \[i_X: X \to X''\] For $x \in X$ we define the \underline{canonical map} \[i_x: X \to X''\] by \[i_X(x)(T) = Tx \text{ for } T \in X'\] $i_X$ is well-defined and $i_X(x)$ is an element of the bidual space. Moreover, the map is injective and an isometry. Spaces $X$ where $i_X$ is also surjective are also called \underline{reflexive spaces}.
	\end{definition}
	\begin{example}
		All Hilbert spaces, $L^p$ spaces and all Sobolev spaces are reflexive for $1<p<\infty$.
	\end{example}
	\begin{definition}
		Let $(x_k)_{k \in \mathbb{N}} \subset X$. $x_k$ converges weakly to $x \in X$ if for all $T \in X'$ \[T(x_k) \to T(x) \text{ for } k \to \infty\] 
	\end{definition}
	\begin{theorem}
		Every bounded sequence in a reflexive space admits a weakly convergent subsequence.
	\end{theorem}
	\begin{theorem}
		Let $(X, ||\cdot ||)$ be a reflexive Banach space. Let $M\subset X$ and $M\neq \varnothing$ a weakly sequentially closed subset $X$. Let $E:M \to \mathbb{R}$ s.t. \begin{enumerate}
			\item $E(y) \to \infty$ if $||y||_X \to \infty$
			\item $E$ is sequentially weakly lower semi-continuous that is if $x_k$ converges weakly to $x \in X$, than $E(x) \leq \liminf_{k \to \infty} E(x_k)$
		\end{enumerate}
		Then, $E$ is bounded from below and $\exists x \in M$ s.t. \[E(x) = \inf\{E(y): \; y \in M\}\]
	\end{theorem}
\section{A partition problem and functions of bounded variation}
	Given $\Omega \subset \mathbb{R}^n$ bounded and smooth, is there an $E \subseteq \Omega$ s.t. \[\left|E\right| = \left|\Omega \setminus E\right|\]
	and $H^{n-1}(\partial E\cap \Omega)$ is minimal?
	\subsection{Motivation}
	\begin{definition}
		If $E \subset \Omega$ is smooth, then \[H^{n-1}(\partial E \cap \Omega) = \int_{\partial E \cap \Omega} 1 dS(x) = \int_{\partial E \cap \Omega} \langle v(x), v(x) \rangle dS(x)\]
		Let $\varphi \in C:0^\infty(\Omega; \mathbb{R}^n)$ s.t. $\varphi(x) = \lambda(x)v(x)$ where $\lambda(x) \in [0,1]$ and $x \in \partial E$. Then $$H^{n-1}(\partial E \cap \Omega) \geq \int_{\partial E \cap \Omega} \langle \varphi(x), v(x)\rangle dS(x) = \int_{\partial E} \langle \varphi(x), v(x)\rangle dS(x) = \int_\Omega \chi_E div(\varphi(x))dx$$
		Since $D$ is smooth, $\exists \psi: \mathbb{R}^n \to \mathbb{R}^n$ with \begin{itemize}
			\item $\psi(x) = v(x)$ on $\partial E \cap \Omega$
			\item $\psi(x) = 0$ ``some bit away from $\Omega$''
			\item $||\psi(x)||_2 \leq 1 \; \forall x \in \mathbb{R}^n$ and $||\psi||_\infty \leq 1$.
		\end{itemize}
		Now, let $\xi_\varepsilon$ be the bump function on $\Omega$ which converges to $\chi_\Omega$ for $\varepsilon \to 0$. Then $\varphi_\varepsilon = \xi_\varepsilon \psi$ satisfies $\varphi_\varepsilon \in C_0^\infty(\Omega, \mathbb{R}^n)$ and $||\varphi_\varepsilon||_\infty \leq 1$. Now \[H^{n-1}(\partial E \cap \Omega) \overset{\varepsilon \searrow 0}{\to} \int_{\partial E \cap \Omega} dS(x) = H^{n-1}(\partial E \cap \Omega)\]
		Hence \[H^{n-1}(\partial E \cap \Omega) = \sup\{\int_\Omega \chi_E div \varphi dx \; | \; \varphi \in C_0^\infty(\Omega, \mathbb{R}^n), \; ||\varphi||_\infty \leq 1\}\]
		The idea is to use this expression for non-smooth $E$.
	\end{definition}
	\subsection{Functions of bounded variations}
	Let $\Omega \subset \mathbb{R}^n$ be open and bounded. 
	\begin{definition}
		We define for $v \in L^1(\Omega)$
		$$\int_\Omega \left|Dv\right| = \sup\{\int_\Omega v(x) div \varphi(x) dx \; | \; \varphi \in C_0^\infty(\Omega; \mathbb{R}^n), \; ||\varphi||_\infty \leq 1\}$$ the total variation of $v$.\\
		We say that $v$ is of bounded variation, if \[\int \left|Dv\right| < \infty\]
		Finally, we define $BV(\Omega) = \{v \in L^1(\Omega): \; \int_\Omega \left|Dv\right| < \infty\}$
	\end{definition}
	\begin{remark}
		It holds that $W^{1,1}(\Omega) \subset BV(\Omega)$. For $v \in W^{1,1}(\Omega)$ it holds that \[\int_\Omega \left|Dv\right| = \int_\Omega\left|\nabla v\right| \; dx\] Finally, above inclusion is even strict: $W^{1,1}(\Omega) \subsetneq BV(\Omega)$
	\end{remark}
	\begin{lemma}
		Let $\Omega$ be a bounded domain. Define $||\cdot ||_{BV(\Omega)}: BV(\Omega) \to \mathbb{R}$ via \[||u||_{BV(\Omega)} = ||u||_{L^1} + \int_\Omega \left|Du\right|\] This is a norm and $(BV(\Omega), ||\cdot||_{BV(\Omega)})$ is a Banach space.
	\end{lemma}
	\begin{definition}
		Let $E\subset \mathbb{R}^n$ be measurable. \[P(E, \Omega) = \int_\Omega \left|D\chi_{E\cap \Omega}\right|\] is called the perimeter of $E$ in $\Omega$. 
	\end{definition}
	\begin{remark}
		If $\partial E$ is smooth, then $P(E,\Omega) = H^{n-1}(\partial E \cap \Omega)$
	\end{remark}
	\noindent\underline{QUESTION:} For $\Omega \subset \mathbb{R}^n$ bounded domain, define \[M = \{E \subset \Omega \mid \lambda^n(E) = \lambda^n(\Omega\setminus E) = \frac{1}{2} \lambda^n(\Omega)\}\] Is \[\inf \{P(E,\Omega) \mid E \in M\}\] attained?\\
	\underline{IDEA:} Take $(E_n)_{n \in \mathbb{N}} \subset \Omega$ a minimizing sequence, that is \[P(E_n, \Omega) = \int_\Omega \left|D\chi_{E_n \cap \Omega}\right| \;\searrow\; \inf\{P(E,\Omega) \mid E \in M\}\]
	\subsection{Lower semicontinuity in $BV(\Omega)$}
	\begin{theorem}
		Let $\Omega$ be a bounded domain. Let $(v_k)_{k \in \mathbb{N}} \subset BV(\Omega)$ s.t. $v_k \to v$ in $L^(\Omega)$. Then \[\int_\Omega \left|Dv\right| \leq \liminf \int_\Omega \left|Dv_k\right|\] In particular, if $\liminf_{k \to \infty} \int_\Omega \left|Dv_k\right| < \infty$ then $v \in BV(\Omega)$.
	\end{theorem}
	\subsection{Lipschitz continuity}
	\begin{definition}
		A function $v: \bar{\Omega} \to \mathbb{R}$ is called Lipschitz continuous if \[[v]_{0,1} = \sup_{x,y \in \Omega, x \neq y} \frac{\left|v(x)-v(y)\right|}{\left|x-y\right|} < \infty\]
		Using this, one defines $Lip(\Omega) = \{v: \bar{\Omega} \to \mathbb{R}: v \text{ is Lipschitz}\} = C^{0,1}(\Omega)$.\\
		$Lip_{loc}(\Omega) = \{v: \Omega \to \mathbb{R}: v \in Lip(K)\;  \forall K \subset \Omega \text{ compact}\}$
	\end{definition}
	\begin{lemma}
		$||\cdot||_{Lip}: Lip(\Omega) \to \mathbb{R}$ defined by \[||u||_{Lip} = [u]_{0,1} + ||u||_{C^0}\] makes $Lip(\Omega)$ a Banach space.
	\end{lemma}
	\begin{theorem}
		Let $v \in Lip(\mathbb{R}^n)$. Then $v$ is a.e. classically differentiable. If $\Omega$ is a domain, then $v \in Lip_{loc}(\Omega) \Leftrightarrow v \in W_{loc}^{1,\infty}(\Omega)$. If $\partial \Omega$ is $C^1$, then $v \in Lip(\Omega) \Leftrightarrow v \in W^{1,\infty}(\Omega)$. In this case $[v]_{0,1} = ||\nabla v||_{L^\infty}$.
	\end{theorem}
	\begin{definition}
		$Lip(\Omega, g) = \{u \in Lip(\Omega) \mid u \equiv g \text{ on } \partial \Omega\}$\\
		$Lip(\Omega, g, K) = \{u \in Lip(\Omega,g) \mid [u]_{0,1} \leq K\}$
	\end{definition}
	The aim is now to minimize $F$ in $Lip(\Omega, g)$ where \[F(u) = \int_\Omega \sqrt{1 + \left|\nabla u\right|^2} \, dx\]
	\underline{Method}: Solve minimization in $Lip(\Omega, g, K)$ and then show that the minimizer satisfies $[u]_{0,1} < K$ strictly for a large enough $K$ and conclude that $u$ minimizes $F$ in $Lip(\Omega, g)$.
	\begin{lemma}
		Let $\Omega$ be a bounded $C^1$-domain, $g \in Lip(\Omega)$ with $[g]_{0,1} \leq K$. Then $\exists u \in Lip(\Omega, g, K)$ s.t. \[F(u) \leq F(v) \; \forall v \in Lip(\Omega, g, K)\]
	\end{lemma}
	\begin{lemma}
		Let $u^k$ be a minimizer of $F$ in $Lip(\Omega, g, k)$. If $[u^k]_{0,1} <k$ then $u^k$ minimizes $F$ in $Lip(\Omega,g)$.
	\end{lemma}
	\begin{definition}
		Let $v \in Lip(\Omega,K)$. We call $v$ a \underline{superminimum} (resp. \underline{subminimum}) for $F$ if $\forall w \in Lip(\Omega, v, K)$ s.t. $w \geq v$ in $\Omega$ then $F(v) \leq F(w)$ (resp. $w\leq v$).
	\end{definition}
	\begin{theorem}[Maximum principle]
		Let $\Omega \subset \mathbb{R}^n$ be open and bounded. Let $v \in Lip(\Omega,K)$ be a superminimum, $w \in Lip(\Omega,K)$ be a subminimum s.t. $v \geq w$ on $\partial \Omega$. Then $v \geq w$ in $\Omega$.
	\end{theorem}
	\begin{corollary}
		Let $\Omega\subset \mathbb{R}^n$ open and bounded. Let $v \in Lip(\Omega, K)$ be a superminimum and $w \in Lip(\Omega, K)$ a subminimum. Then \[\sup_{\partial \Omega} w-v = \sup_{\Omega} w-v\]
	\end{corollary}
	\begin{corollary}
		If $u, \tilde{u} \in Lip(\Omega,g,K)$ are minima of $F$ in $Lip(\Omega,f,K)$ then $u \equiv \tilde{u}$.
	\end{corollary}
	\underline{Goal}: Prove that $u^k$ satisfies \[\sup_{x,y \in \Omega, x \neq y} \frac{\left|u^k(x) - u^k(y)\right|}{\left|x-y\right|} = [u^k]_{0,1} < K\]
	\begin{lemma}
		Let $\Omega \subset \mathbb{R}^n$ bounded and open. Let $u \in Lip(\Omega,g,K)$ be a minimum of $F$. Then \[[u]_{0,1} = \sup_{x \in \Omega, y \in \partial \Omega, x \neq y} \frac{\left|u(x)-u(y)\right|}{\left|x-y\right|}\]
	\end{lemma}
	\begin{lemma}
		Let $a \in \mathbb{R}$, $z \in \mathbb{R}^n$ and $w:\mathbb{R}^n \to \mathbb{R}$ via $w(c) = a+zx$. Then $w$ minimizes $F$ in $Lip(\Omega,w)$. 
	\end{lemma}
	\section{Obstacle problems}
	Let $\Omega \subset \mathbb{R}^n$ be a bounded convex domain with $\partial \Omega \in C^1$. Let $h \in Lip(\Omega)$ s.t. $h|_{\partial \Omega} <0$ and $h>0$ somewhere in $\Omega$.\\
	\underline{Problem}: Is $\inf\{F(u): u \in Lip(\Omega), u|_{\partial \Omega} = 0, u \geq h\}$ attained? As before $F(u) = \int_\Omega \sqrt{1+\left|\nabla u\right|^2}$. That is, does a $u \in M = \{v \in Lip(\Omega): v|_{\partial \Omega} = 0, v \geq h\}$ s.t. \[F(u) \leq F(v) \; \forall v \in M\] exist?\\
	We call $h$ the obstacle.\\
	The difference of this problem to the previous problems is in the first variation. If $u$ is a minimizer in the previous problems, then $\frac{d}{dt} F(u+t\varphi)|_{t=0} = 0$ $\forall \varphi \in C_0^\infty(\Omega)$. Here we have only information form some directions. Since $M$ is convex for all $v \in M$ $tu+(1-t)v$ is again in $M$ for $t \in [0,1]$. Hence, if $u$ is a minimizer, $t \mapsto F(tu+(1-t)v)$ has a minimum at $t=1$. Hence \[\frac{d}{dt} F(tu+(1-t)v)|_{t=1} = \int_\Omega \frac{\langle\nabla(tu+(1-t)v),\nabla(u-v)\rangle}{1+\left|\nabla(tu + (1-t)v)\right|^2} \, dx |_{t=1}\leq 0\]
	This can be rearranged to \[\int_\Omega\frac{\nabla u}{\sqrt{1+\left|\nabla u\right|^2}} \nabla(u-v) \, dx \geq 0 \, \forall v \in M\] This is called a variational inequality.\\
	The strategy to show the existence of a minimizer consists of the following steps
	\begin{enumerate}
		\item add another constraint $[u]_{0,1} \leq k$
		\item if minimizer satisfies this with strict inequality then it is a minimizer without the additional constraint
	\end{enumerate}
	Let $M^k = \{u \in M: [u]_{0,1} \leq k\}$ for $k > [h]_{0,1}$ and note that $\max\{0,h\} \in M^k$. Notice that $M^k$ is also convex.
	\begin{lemma}
		$\exists u \in M^k$ s.t. $F(u) \leq F(v) \; \forall v \in M^k$.
	\end{lemma}
	\begin{lemma}
		Let $u^k$ be a minimizer for $F$ in $M^k$ s.t. $[u^k]_{0,1} <k$. Then $u^k$ minimizes $F$ in $M$.
	\end{lemma}
	Our aim is now to find an a-priori estimate for solutions of variational inequalities of the following type \[\int_\Omega a(\nabla u) \cdot (\nabla(v-u))\, dx \geq 0 \, \forall v \in M \cup M^k\] where $a$ satisfies the following strong ellipticity condition \[(a(p)-a(q)) \cdot (p-q) > 0 \, \forall penis,q \in \mathbb{R}^n, p \neq q\]
	\begin{theorem}
		Let $u \in M^k$ be a solution of \[\int_\Omega a(\nabla u) \cdot \nabla(v-u) \, dx \geq 0 \; \forall v \in M^k\] where $a$ satisfies the strong ellipticity. Then \[\sup_{x\neq y \in \Omega} \frac{\left|u(x)-u(y)\right|}{\left|x-y\right|} \leq \max\left\{[h]_{0,1}, \sup_{x \in \Omega. y \in \partial \Omega} \frac{\left|u(x)-u(y)\right|}{\left|x-y\right|}\right\}\]
	\end{theorem}
	\begin{theorem}
		Let $\Omega \subset \mathbb{R}^n$ be a convex, bounded domain with $\partial \Omega \in C^1$. Let $a:\mathbb{R}^n \to \mathbb{R}^n$ satisfy the strong ellipticity and $h \in Lip(\Omega)$, $h|_{\partial \Omega} <0$ but $h>0$ somewhere in $\Omega$. Let $k > [h]_{0,1}$ and $u \in M^k$ be a solution of \[\int_\Omega a(\nabla u)\cdot \nabla(v-u) dx \geq 0 \; \forall v \in M^k\]
		Then \begin{enumerate}
			\item $u \geq 0$
			\item $u$ is unique
			\item $[u]_{0,1} \leq [h]_{0,1} < k$
		\end{enumerate}
	\end{theorem}
	\begin{theorem}
		Let $\Omega \subset \mathbb{R}^n$ be a bounded convex domain with $\partial \Omega \in C^1$. Let $h \in Lip(\Omega)$, $h|_{\partial \Omega} <0$ and $h>0$ somewhere in $\Omega$. Then $\exists! u \in Lip(\Omega)$, $u|_{\partial \Omega} = 0$ and $u \geq h$ in $\Omega$. \[F(u) \leq F(v) \forall v \in Lip(\Omega), \; v|_{\partial \Omega) = 0}\]
		where $F(u) = \int_\Omega \sqrt{1+\left|\nabla u\right|^2} \, dx$. Moreover, $u$ solves \[\int_\Omega \frac{\nabla u}{1+\left|\nabla u\right|^2} \nabla (v-u)\, dx \geq 0\] and $u$ satisfies $[u]_{0,1} \leq [h]_{0,1}$ and $||u||_{L^\infty} \leq [h]_{0,1} diam(\Omega)$.\\
		Moreover, define $I = \{x \in \Omega: u(x) = h(x)\}$ the coincidence set, then \[\int_\Omega \frac{\nabla u}{\sqrt{1+\left|\nabla u\right|^2}} \nabla \varphi \, dx = 0 \; \forall \varphi \in Lip(\Omega), \, supp(\varphi) \subset \Omega \setminus I\]
	\end{theorem}
	\begin{remark}
		Is $I\neq \varnothing$? Is it connected and what is its measure? 
	\end{remark}
	\section{The Schrödinger equation}
	We describe an electron in a universe with only one nucleus that attracts the electron. Motivated by the uncertainty principle of Heisenberg, the electron is described by a function $\psi: \mathbb{R}^n \to \mathbb{C}$, $||\psi||_{L^2(\mathbb{R}^n)}=1$ a wave function.\\
	Idea: $\Omega \subset \mathbb{R}^n$ open, $\int_\Omega \left|\psi(x)\right|^2 \, dx$ is the probability that the electron is in $\Omega$.\\
	We consider $n \geq 3$. Define \begin{itemize}
		\item \underline{Kinetic energy}: $\int_{\mathbb{R}^n} \left|\nabla \psi\right|^2 \, dx = T(\psi)$
		\item \underline{potential energy}: $V(\psi) = \int_{\mathbb{R}^n} v(x) \left|\psi(x)\right|^2 \, dx$. Typical choice would be $n=3$ (for obvious reasons), $v(x) = -\frac{z}{\left|x\right|}$ with a nucleus of charge $z$ at the origin.
		\item \underline{Total energy}: $E(\psi) = T(\psi) + V(\psi)$
	\end{itemize}
	\underline{Question:} Is \[E_0 = \inf\{E/\psi): \psi \in W^{1,2}(\mathbb{R}^n), \; ||\psi||_{L^2(\mathbb{R}^n)} = 1\}\] attained? $E_0$ is called the ground state energy. If a minimum exists, it is called ground state.
	\begin{definition}
		Let $f:\mathbb{R}^n \to \mathbb{C}$ then we say that $f$ \underline{vanishes at $\infty$} if $\forall a > 0$ \[\lambda^n(\{x \in \mathbb{R}^n: \left|f(x)\right| > a\}) < \infty\]
		We define \[D^1(\mathbb{R}^n) = \{f: \mathbb{R}^n \to \mathbb{C} \mid \nabla f \in L^2(\mathbb{R}^n), \text{ and $f$ vanishes at } \infty\}\]
	\end{definition}
	\begin{theorem}
		Let $n\geq 3$. Then $\exists S_n > 0$ s.t. $\forall f\in D^1(\mathbb{R}^n)$ \[S_n||f||_{L^\frac{2n}{n-2}(\mathbb{R}^n)} \leq ||\nabla f||_{L^2(\mathbb{R}^n)}\]
	\end{theorem}
	\begin{remark}
		Let $n \geq 3$, $V \in L^{\frac{n}{2}} + L^\infty(\mathbb{R}^n)$. That means $v=v_1+v_2$ where $v_1 \in L^{\frac{n}{2}}$, $v_2 \in L^\infty(\mathbb{R}^n)$. Then $\exists \lambda \in \mathbb{R}$ s.t. $\forall \psi \in W^{1,2}(\mathbb{R}^n)$ with $||\psi||_{L^2} = 1$ \[E(\psi) \geq \frac{1}{2} \int_{\mathbb{R}^n} \left|\nabla \psi\right|^2 - \lambda - ||v_2||_{L^2} \geq -\lambda - ||v_2||_{L^2}\]
	\end{remark}
	Using this result, we can conclude that the potential energy is even continuous. That is \[\int_{\mathbb{R}^n} V(x)\left|\psi_j(x)\right|^2 \, dx \to \int_{\mathbb{R}^n} V(x) \left|\psi_0(x)\right|^2 \, dx\] for a convergent subsequence of a minimizing sequence $\psi_j$.
	\begin{lemma}
		If $f_k \to f$ weakly in $W^{1,2}(\mathbb{R}^n)$, let $A\subset \mathbb{R}^n$ measurable s.t. $\lambda^n(A) < \infty$. Then \[\chi_A f_k \to \chi_A f \text{ in } L^{p}(\mathbb{R}^n)\] for $1\leq p \leq \frac{2n}{n-2}$.
	\end{lemma}
	\begin{lemma}
		Under the same assumptions as in the previous Lemma, and the additional one that $v$ vanishes at $\infty$, i.e. $\forall a >0$ it holds that $\lambda^n(\{\left|V(x)\right|>a\}) < \infty$, then if $\psi_j \to \psi$ weakly in $W^{1,2}(\mathbb{R}^n)$, $||\psi_j||_{L^2} = 1 \; \forall j$ then \[\int_{\mathbb{R}^n} V(x) \left|\psi_j(x)\right|^2 \, dx \to \int_{\mathbb{R}^n} V(x) \left|\psi(x)\right|^2 \, dx\]
	\end{lemma}
\end{document}