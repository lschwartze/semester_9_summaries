\documentclass[a4paper, 12pt]{article}

\usepackage{fullpage}
\usepackage[utf8]{inputenc}
\usepackage[english]{babel}
\usepackage{amsmath,amssymb}
\usepackage[explicit]{titlesec}
\usepackage{ulem}
\usepackage[onehalfspacing]{setspace}
\usepackage{amsthm}

\theoremstyle{plain}
\newtheorem{theorem}{Theorem}[section] % reset theorem numbering for each chapter

\theoremstyle{definition}
\newtheorem{definition}[theorem]{Definition} % definition numbers are dependent on theorem numbers
\theoremstyle{lemma}
\newtheorem{lemma}[theorem]{Lemma}

\theoremstyle{remark}
\newtheorem{remark}[theorem]{Remark}

\theoremstyle{corollary}
\newtheorem{corollary}[theorem]{Corollary}

\theoremstyle{example}
\newtheorem{example}[theorem]{Example}

\titleformat{\subsection}
{\small}{\thesubsection}{1em}{\uline{#1}}
\begin{document}
	\begin{titlepage} 
		\title{Calculus of Variations}
		\clearpage\maketitle
		\thispagestyle{empty}
	\end{titlepage}
	\tableofcontents
	\newpage
	\section{Introduction}
	The objects of calculus of variations are so-called functionals, i.e. functions of functions. The main interests are the critical points of these functionals. Since such critical points are often solutions to (elliptic) PDEs. We will therefore look for minima and saddle points.
	\begin{example}[soap bubbles]
		Which object encloses a fixed volume and has smallest surface area?\\
		Formulation: \[\mathcal{F}: \{\text{surfaces enclosing volume } v_0\} \to \mathbb{R}_{\geq 0}\]
		The minimizer is a sphere.
	\end{example}
	\subsection{mathematical formulation}
	Let $X$ be a Banach space. Let $\varnothing \neq U \subset X$ and \[E: U \to \mathbb{R}\] be a functional.\\
	Now, if $E$ is bounded from below (i.e. $\exists M>0: E(u) \geq -M \; \forall u \in U$), is there a $u \in U$ s.t. $E(u) = \inf_{v \in U} E(v)$? Are there other critical points?\\
	There are two ways to approach the first question.
	\begin{enumerate}
		\item if $U$ is open and $E$ is differentiable then look for a solution to \[E'(u) = 0\] (called the classical method in CV). This will lead us to $u$ being a solution to a PDE.
		\item take a sequence $(u_n)_{n \in \mathbb{N}} \subset U$ s.t. $E(u_n) \to \inf_{v \in U} E(v)$ (a minimizing sequence).
		\begin{enumerate}
			\item Find a topology s.t. $\exists (u_{n_k})_{k \in \mathbb{N}}, u \in U$ with \[u_{n_k} \to u\] in this topology.
			\item check if $E(u) \leq \liminf_{k \to \infty} E(u_{n_k})$.
		\end{enumerate}
	\end{enumerate}
	\section{First variation and convexity}
	\subsection{First variation}
	We fix $X$ to be a Banach space
	\begin{definition}
		Let $\varnothing \neq V\subset X$ be open. Let $E: V \to \mathbb{R}$. We say that $E$ is \underline{Fréchet differentiable} at $u \in V$ if $\exists A \in L(X,\mathbb{R})$ (a linear bounded map) s.t. \[\lim_{||\varphi||_x \to 0} \frac{E(u + \varphi) - E(u) - A\varphi}{||\varphi||_x} = 0\]
	\end{definition}
	$A$ is called the Fréchet derivative of $E$ at $u$ denoted by $E'(u)$.
	\begin{definition}[First variation]
		Let $\varnothing \neq V \subset X$, $E:V \to \mathbb{R}$ and $u \in V$. Let $\varphi \in X$ s.t. \[\exists \delta > 0 \text{ s.t. } u + t\varphi \in V \; \forall t \in (-\delta, \delta)\]
		If \[(-\delta, \delta) \ni t \mapsto E(u+t\varphi) \in \mathbb{R}\] is differentiable at $t=0$ we say that $E$ has first in variation $u$ in direction $\varphi$ and write \[\delta E(u)(\varphi) = \frac{d}{dt} E(u+t\varphi)|_{t = 0}\]
	\end{definition}
	\begin{theorem}[Fundamental Lemma of CV]
		Let $\Omega \subset \mathbb{R}^n$ open, $w \in L_{loc}^1(\Omega)$ (i.e. $\forall K \subset \Omega$ compact, $w \in L^1(K))$. If \[\int_\Omega w\varphi \,dx = 0 \; \forall \varphi \in C_0^\infty(\Omega)\] then $w = 0$ a.e.
	\end{theorem}
	\begin{corollary}
		Let $n \in \mathbb{N}$, $u \in L^1_{loc}((a,b))$ s.t. \[\int_a^b u \frac{d^n}{dx^n} \varphi \,dx = 0 \,\forall \varphi \in c_0^\infty ((a,b))\] Then $\exists a_1,...,a_n \in \mathbb{R}$ s.t. \[u(x) = \sum_{i=0}^{n-1} a_i x^i\]
	\end{corollary}
	\begin{definition}
		Let $u \in L^1_{loc}(\Omega)$. We say that $u$ is once weakly differentiable if $\forall i \in \{1,2,...,n\}$ $\exists v_i \in L^1_{loc}(\Omega)$ s.t. \[\int_\Omega u \partial_{x_i}\varphi \, dx = - \int_\Omega v_i \varphi \,dx \; \forall \varphi \in C_0^\infty (\Omega)\] In that case $\partial_i u = v_i$.\\
		Similarly, if $\alpha \in \mathbb{N}_0^n$ we say $u$ is $\alpha$-weakly differentiable if \[\exists v_\alpha \in L^1_{loc} \text{ s.t. } \int_\Omega uD^\alpha \varphi \, dx = (-1)^{\left|\alpha\right|} \int v_\alpha \varphi \, dx\]
	\end{definition}
	\begin{definition}[Sobolev spaces]
		For $m \in \mathbb{N}$, $p \in (1,\infty)$, define \[W^{m,p}(\Omega) = \{u \in L^p(\Omega) \text{ s.t. } D^\alpha u \in L^p(\Omega) \; \forall \alpha \in \mathbb{N}_0^n, \; \left|\alpha\right| \leq m\}\] This is a vector space and can be equipped with the norm \[||u||_{W^{m,p}} = \left(\sum_{\left(\alpha\right) \leq m} ||D^\alpha u||^p_{L^p(\Omega)}\right)^{\frac{1}{p}}\]
	\end{definition}
	\begin{theorem}
		$(W^{m,p}(\Omega), ||\cdot||_{W^{m,p}})$ is a Banach space and for $p=2$ it's a Hilbert space. Finally, $C^\infty(\Omega) \cap W^{m,p}(\Omega)$ is dense in $W^{m,p}(\Omega)$.
	\end{theorem}
	\begin{definition}
		$W_0^{m,p}(\Omega) = \overline{C_0^\infty(\Omega)}^{W^{m,p}(\Omega)}$, i.w. $f \in W_0^{m,p}(\Omega)$ if \[\exists (f_n)_{n \in \mathbb{N}} \subset C_0^\infty (\Omega)\] s.t. \[||f-f_n||_{W^{m,p}} \to 0, \, n \to \infty\]
	\end{definition}
	\begin{remark}
		$W_0^{m,p}(\Omega) \subset W^{m,p}(\Omega)$
	\end{remark}
	\begin{definition}[weak solution of Poisson equation]
		We say that $u$ is a \underline{weak solution} of \[\begin{cases}
			-\nabla u &= f, \text{ in } \Omega\\
			u &= 0, \text{ on } \partial \Omega
		\end{cases}\] if $u \in W_0^{1,2}(\Omega)$ and \[\int_\Omega (\nabla u \nabla \varphi - f\varphi) \, dx = 0 \; \forall \varphi \in C_0^\infty(\Omega)\]
	\end{definition}
	\section{Direct Methods}
	\begin{lemma}[Poincaré-inequality]
		$\exists C = C(\Omega)$ s.t. \[||u||_{L^2} \leq C||\nabla u||_{L^2} \; \forall u \in W_0^{1,2}(\Omega)\]
	\end{lemma}
	\begin{lemma}
		Define $||u||_{W_0^{1,2}(\Omega)} = ||\nabla u||_{L^2}$, then $||\cdot ||_{W^{1,2}}$ and $||\cdot ||_{W_0^{1,2}}$ are equivalent in $W_0^{1,2}(\Omega)$.
	\end{lemma}
	\begin{theorem}
		Every bounded sequence in a Hilbert space admits a weakly convergent subsequence. That means given $(v_k)_{k \in \mathbb{N}} \subset H$ bounded in Hilbert space $H$ $\exists (v_{k_l})_{l \in \mathbb{N}} \exists v \in H$ s.t. \[\langle v_{k_l}, w\rangle_H \to \langle v, w\rangle_H \; \forall w \in H\]
	\end{theorem}
	\begin{theorem}
		If $(u_k)_{k \in \mathbb{N}}$ converges weakly to $u$ in $H$, then \[||u||_H \leq \liminf_{k \to \infty} ||u_k||_H\]
	\end{theorem}
	\begin{remark}
		A functional $T:X\to Y$ is compact if $\forall (x_k)_{k \in \mathbb{N}}$ bounded in $X$, the sequence $(Tx_n)_{n \in \mathbb{N}} \subset Y$ admits a convergent subsequence.
	\end{remark}
	\begin{theorem}
		The inclusion map $W_0^{1,2} \rightarrow L^2(\Omega)$ that maps $u \mapsto u$ is compact.
	\end{theorem}
	\subsection{Some concepts from functional analysis}
	Let $X$ be a $\mathbb{R}$-Banach space.\\
	\begin{definition}
		We define the dual space \[X' = L(X,\mathbb{R}) = \{T: X \to \mathbb{R} \text{ linear and continuous}\}\]
	\end{definition}
	\begin{lemma}
		For linear maps, the following are equivalent:
		\begin{enumerate}
			\item continuity
			\item continuity at 0
			\item boundedness
		\end{enumerate}
	\end{lemma}
	\begin{definition}
		Similarly, the bidual space \[X'' = L(X', \mathbb{R})\] is also a Banach space. There exists a canonical map \[i_X: X \to X''\] For $x \in X$ we define the \underline{canonical map} \[i_x: X \to X''\] by \[i_X(x)(T) = Tx \text{ for } T \in X'\] $i_X$ is well-defined and $i_X(x)$ is an element of the bidual space. Moreover, the map is injective and an isometry. Spaces $X$ where $i_X$ is also surjective are also called \underline{reflexive spaces}.
	\end{definition}
	\begin{example}
		All Hilbert spaces, $L^p$ spaces and all Sobolev spaces are reflexive for $1<p<\infty$.
	\end{example}
	\begin{definition}
		Let $(x_k)_{k \in \mathbb{N}} \subset X$. $x_k$ converges weakly to $x \in X$ if for all $T \in X'$ \[T(x_k) \to T(x) \text{ for } k \to \infty\] 
	\end{definition}
	\begin{theorem}
		Every bounded sequence in a reflexive space admits a weakly convergent subsequence.
	\end{theorem}
	\begin{theorem}
		Let $(X, ||\cdot ||)$ be a reflexive Banach space. Let $M\subset X$ and $M\neq \varnothing$ a weakly sequentially closed subset $X$. Let $E:M \to \mathbb{R}$ s.t. \begin{enumerate}
			\item $E(y) \to \infty$ if $||y||_X \to \infty$
			\item $E$ is sequentially weakly lower semi-continuous that is if $x_k$ converges weakly to $x \in X$, than $E(x) \leq \liminf_{k \to \infty} E(x_k)$
		\end{enumerate}
		Then, $E$ is bounded from below and $\exists x \in M$ s.t. \[E(x) = \inf\{E(y): \; y \in M\}\]
	\end{theorem}
\section{A partition problem and functions of bounded variation}
	Given $\Omega \subset \mathbb{R}^n$ bounded and smooth, is there an $E \subseteq \Omega$ s.t. \[\left|E\right| = \left|\Omega \setminus E\right|\]
	and $H^{n-1}(\partial E\cap \Omega)$ is minimal?
	\subsection{Motivation}
	\begin{definition}
		If $E \subset \Omega$ is smooth, then \[H^{n-1}(\partial E \cap \Omega) = \int_{\partial E \cap \Omega} 1 dS(x) = \int_{\partial E \cap \Omega} \langle v(x), v(x) \rangle dS(x)\]
		Let $\varphi \in C:0^\infty(\Omega; \mathbb{R}^n)$ s.t. $\varphi(x) = \lambda(x)v(x)$ where $\lambda(x) \in [0,1]$ and $x \in \partial E$. Then $$H^{n-1}(\partial E \cap \Omega) \geq \int_{\partial E \cap \Omega} \langle \varphi(x), v(x)\rangle dS(x) = \int_{\partial E} \langle \varphi(x), v(x)\rangle dS(x) = \int_\Omega \chi_E div(\varphi(x))dx$$
		Since $D$ is smooth, $\exists \psi: \mathbb{R}^n \to \mathbb{R}^n$ with \begin{itemize}
			\item $\psi(x) = v(x)$ on $\partial E \cap \Omega$
			\item $\psi(x) = 0$ ``some bit away from $\Omega$''
			\item $||\psi(x)||_2 \leq 1 \; \forall x \in \mathbb{R}^n$ and $||\psi||_\infty \leq 1$.
		\end{itemize}
		Now, let $\xi_\varepsilon$ be the bump function on $\Omega$ which converges to $\chi_\Omega$ for $\varepsilon \to 0$. Then $\varphi_\varepsilon = \xi_\varepsilon \psi$ satisfies $\varphi_\varepsilon \in C_0^\infty(\Omega, \mathbb{R}^n)$ and $||\varphi_\varepsilon||_\infty \leq 1$. Now \[H^{n-1}(\partial E \cap \Omega) \overset{\varepsilon \searrow 0}{\to} \int_{\partial E \cap \Omega} dS(x) = H^{n-1}(\partial E \cap \Omega)\]
		Hence \[H^{n-1}(\partial E \cap \Omega) = \sup\{\int_\Omega \chi_E div \varphi dx \; | \; \varphi \in C_0^\infty(\Omega, \mathbb{R}^n), \; ||\varphi||_\infty \leq 1\}\]
		The idea is to use this expression for non-smooth $E$.
	\end{definition}
	\subsection{Functions of bounded variations}
	Let $\Omega \subset \mathbb{R}^n$ be open and bounded. 
	\begin{definition}
		We define for $v \in L^1(\Omega)$
		$$\int_\Omega \left|Dv\right| = \sup\{\int_\Omega v(x) div \varphi(x) dx \; | \; \varphi \in C_0^\infty(\Omega; \mathbb{R}^n), \; ||\varphi||_\infty \leq 1\}$$ the total variation of $v$.\\
		We say that $v$ is of bounded variation, if \[\int \left|Dv\right| < \infty\]
		Finally, we define $BV(\Omega) = \{v \in L^1(\Omega): \; \int_\Omega \left|Dv\right| < \infty\}$
	\end{definition}
	\begin{remark}
		It holds that $W^{1,1}(\Omega) \subset BV(\Omega)$. For $v \in W^{1,1}(\Omega)$ it holds that \[\int_\Omega \left|Dv\right| = \int_\Omega\left|\nabla v\right| \; dx\] Finally, above inclusion is even strict: $W^{1,1}(\Omega) \subsetneq BV(\Omega)$
	\end{remark}
	\begin{lemma}
		Let $\Omega$ be a bounded domain. Define $||\cdot ||_{BV(\Omega)}: BV(\Omega) \to \mathbb{R}$ via \[||u||_{BV(\Omega)} = ||u||_{L^1} + \int_\Omega \left|Du\right|\] This is a norm and $(BV(\Omega), ||\cdot||_{BV(\Omega)})$ is a Banach space.
	\end{lemma}
	\begin{definition}
		Let $E\subset \mathbb{R}^n$ be measurable. \[P(E, \Omega) = \int_\Omega \left|D\chi_{E\cap \Omega}\right|\] is called the perimeter of $E$ in $\Omega$. 
	\end{definition}
	\begin{remark}
		If $\partial E$ is smooth, then $P(E,\Omega) = H^{n-1}(\partial E \cap \Omega)$
	\end{remark}
	\noindent\underline{QUESTION:} For $\Omega \subset \mathbb{R}^n$ bounded domain, define \[M = \{E \subset \Omega \mid \lambda^n(E) = \lambda^n(\Omega\setminus E) = \frac{1}{2} \lambda^n(\Omega)\}\] Is \[\inf \{P(E,\Omega) \mid E \in M\}\] attained?\\
	\underline{IDEA:} Take $(E_n)_{n \in \mathbb{N}} \subset \Omega$ a minimizing sequence, that is \[P(E_n, \Omega) = \int_\Omega \left|D\chi_{E_n \cap \Omega}\right| \;\searrow\; \inf\{P(E,\Omega) \mid E \in M\}\]
	\subsection{Lower semicontinuity in $BV(\Omega)$}
	\begin{theorem}
		Let $\Omega$ be a bounded domain. Let $(v_k)_{k \in \mathbb{N}} \subset BV(\Omega)$ s.t. $v_k \to v$ in $L^(\Omega)$. Then \[\int_\Omega \left|Dv\right| \leq \liminf \int_\Omega \left|Dv_k\right|\] In particular, if $\liminf_{k \to \infty} \int_\Omega \left|Dv_k\right| < \infty$ then $v \in BV(\Omega)$.
	\end{theorem}
	\subsection{Lipschitz continuity}
	\begin{definition}
		A function $v: \bar{\Omega} \to \mathbb{R}$ is called Lipschitz continuous if \[[v]_{0,1} = \sup_{x,y \in \Omega, x \neq y} \frac{\left|v(x)-v(y)\right|}{\left|x-y\right|} < \infty\]
		Using this, one defines $Lip(\Omega) = \{v: \bar{\Omega} \to \mathbb{R}: v \text{ is Lipschitz}\} = C^{0,1}(\Omega)$.\\
		$Lip_{loc}(\Omega) = \{v: \Omega \to \mathbb{R}: v \in Lip(K)\;  \forall K \subset \Omega \text{ compact}\}$
	\end{definition}
	\begin{lemma}
		$||\cdot||_{Lip}: Lip(\Omega) \to \mathbb{R}$ defined by \[||u||_{Lip} = [u]_{0,1} + ||u||_{C^0}\] makes $Lip(\Omega)$ a Banach space.
	\end{lemma}
	\begin{theorem}
		Let $v \in Lip(\mathbb{R}^n)$. Then $v$ is a.e. classically differentiable. If $\Omega$ is a domain, then $v \in Lip_{loc}(\Omega) \Leftrightarrow v \in W_{loc}^{1,\infty}(\Omega)$. If $\partial \Omega$ is $C^1$, then $v \in Lip(\Omega) \Leftrightarrow v \in W^{1,\infty}(\Omega)$. In this case $[v]_{0,1} = ||\nabla v||_{L^\infty}$.
	\end{theorem}
	\begin{definition}
		$Lip(\Omega, g) = \{u \in Lip(\Omega) \mid u \equiv g \text{ on } \partial \Omega\}$\\
		$Lip(\Omega, g, K) = \{u \in Lip(\Omega,g) \mid [u]_{0,1} \leq K\}$
	\end{definition}
	The aim is now to minimize $F$ in $Lip(\Omega, g)$ where \[F(u) = \int_\Omega \sqrt{1 + \left|\nabla u\right|^2} \, dx\]
	\underline{Method}: Solve minimization in $Lip(\Omega, g, K)$ and then show that the minimizer satisfies $[u]_{0,1} < K$ strictly for a large enough $K$ and conclude that $u$ minimizes $F$ in $Lip(\Omega, g)$.
	\begin{lemma}
		Let $\Omega$ be a bounded $C^1$-domain, $g \in Lip(\Omega)$ with $[g]_{0,1} \leq K$. Then $\exists u \in Lip(\Omega, g, K)$ s.t. \[F(u) \leq F(v) \; \forall v \in Lip(\Omega, g, K)\]
	\end{lemma}
	\begin{lemma}
		Let $u^k$ be a minimizer of $F$ in $Lip(\Omega, g, k)$. If $[u^k]_{0,1} <k$ then $u^k$ minimizes $F$ in $Lip(\Omega,g)$.
	\end{lemma}
	\begin{definition}
		Let $v \in Lip(\Omega,K)$. We call $v$ a \underline{superminimum} (resp. \underline{subminimum}) for $F$ if $\forall w \in Lip(\Omega, v, K)$ s.t. $w \geq v$ in $\Omega$ then $F(v) \leq F(w)$ (resp. $w\leq v$).
	\end{definition}
	\begin{theorem}[Maximum principle]
		Let $\Omega \subset \mathbb{R}^n$ be open and bounded. Let $v \in Lip(\Omega,K)$ be a superminimum, $w \in Lip(\Omega,K)$ be a subminimum s.t. $v \geq w$ on $\partial \Omega$. Then $v \geq w$ in $\Omega$.
	\end{theorem}
	\begin{corollary}
		Let $\Omega\subset \mathbb{R}^n$ open and bounded. Let $v \in Lip(\Omega, K)$ be a superminimum and $w \in Lip(\Omega, K)$ a subminimum. Then \[\sup_{\partial \Omega} w-v = \sup_{\Omega} w-v\]
	\end{corollary}
	\begin{corollary}
		If $u, \tilde{u} \in Lip(\Omega,g,K)$ are minima of $F$ in $Lip(\Omega,f,K)$ then $u \equiv \tilde{u}$.
	\end{corollary}
	\underline{Goal}: Prove that $u^k$ satisfies \[\sup_{x,y \in \Omega, x \neq y} \frac{\left|u^k(x) - u^k(y)\right|}{\left|x-y\right|} = [u^k]_{0,1} < K\]
	\begin{lemma}
		Let $\Omega \subset \mathbb{R}^n$ bounded and open. Let $u \in Lip(\Omega,g,K)$ be a minimum of $F$. Then \[[u]_{0,1} = \sup_{x \in \Omega, y \in \partial \Omega, x \neq y} \frac{\left|u(x)-u(y)\right|}{\left|x-y\right|}\]
	\end{lemma}
	\begin{lemma}
		Let $a \in \mathbb{R}$, $z \in \mathbb{R}^n$ and $w:\mathbb{R}^n \to \mathbb{R}$ via $w(c) = a+zx$. Then $w$ minimizes $F$ in $Lip(\Omega,w)$. 
	\end{lemma}
	\section{Obstacle problems}
	Let $\Omega \subset \mathbb{R}^n$ be a bounded convex domain with $\partial \Omega \in C^1$. Let $h \in Lip(\Omega)$ s.t. $h|_{\partial \Omega} <0$ and $h>0$ somewhere in $\Omega$.\\
	\underline{Problem}: Is $\inf\{F(u): u \in Lip(\Omega), u|_{\partial \Omega} = 0, u \geq h\}$ attained? As before $F(u) = \int_\Omega \sqrt{1+\left|\nabla u\right|^2}$. That is, does a $u \in M = \{v \in Lip(\Omega): v|_{\partial \Omega} = 0, v \geq h\}$ s.t. \[F(u) \leq F(v) \; \forall v \in M\] exist?\\
	We call $h$ the obstacle.\\
	The difference of this problem to the previous problems is in the first variation. If $u$ is a minimizer in the previous problems, then $\frac{d}{dt} F(u+t\varphi)|_{t=0} = 0$ $\forall \varphi \in C_0^\infty(\Omega)$. Here we have only information form some directions. Since $M$ is convex for all $v \in M$ $tu+(1-t)v$ is again in $M$ for $t \in [0,1]$. Hence, if $u$ is a minimizer, $t \mapsto F(tu+(1-t)v)$ has a minimum at $t=1$. Hence \[\frac{d}{dt} F(tu+(1-t)v)|_{t=1} = \int_\Omega \frac{\langle\nabla(tu+(1-t)v),\nabla(u-v)\rangle}{1+\left|\nabla(tu + (1-t)v)\right|^2} \, dx |_{t=1}\leq 0\]
	This can be rearranged to \[\int_\Omega\frac{\nabla u}{\sqrt{1+\left|\nabla u\right|^2}} \nabla(u-v) \, dx \geq 0 \, \forall v \in M\] This is called a variational inequality.\\
	The strategy to show the existence of a minimizer consists of the following steps
	\begin{enumerate}
		\item add another constraint $[u]_{0,1} \leq k$
		\item if minimizer satisfies this with strict inequality then it is a minimizer without the additional constraint
	\end{enumerate}
	Let $M^k = \{u \in M: [u]_{0,1} \leq k\}$ for $k > [h]_{0,1}$ and note that $\max\{0,h\} \in M^k$. Notice that $M^k$ is also convex.
	\begin{lemma}
		$\exists u \in M^k$ s.t. $F(u) \leq F(v) \; \forall v \in M^k$.
	\end{lemma}
	\begin{lemma}
		Let $u^k$ be a minimizer for $F$ in $M^k$ s.t. $[u^k]_{0,1} <k$. Then $u^k$ minimizes $F$ in $M$.
	\end{lemma}
	Our aim is now to find an a-priori estimate for solutions of variational inequalities of the following type \[\int_\Omega a(\nabla u) \cdot (\nabla(v-u))\, dx \geq 0 \, \forall v \in M \cup M^k\] where $a$ satisfies the following strong ellipticity condition \[(a(p)-a(q)) \cdot (p-q) > 0 \, \forall penis,q \in \mathbb{R}^n, p \neq q\]
	\begin{theorem}
		Let $u \in M^k$ be a solution of \[\int_\Omega a(\nabla u) \cdot \nabla(v-u) \, dx \geq 0 \; \forall v \in M^k\] where $a$ satisfies the strong ellipticity. Then \[\sup_{x\neq y \in \Omega} \frac{\left|u(x)-u(y)\right|}{\left|x-y\right|} \leq \max\left\{[h]_{0,1}, \sup_{x \in \Omega. y \in \partial \Omega} \frac{\left|u(x)-u(y)\right|}{\left|x-y\right|}\right\}\]
	\end{theorem}
	\begin{theorem}
		Let $\Omega \subset \mathbb{R}^n$ be a convex, bounded domain with $\partial \Omega \in C^1$. Let $a:\mathbb{R}^n \to \mathbb{R}^n$ satisfy the strong ellipticity and $h \in Lip(\Omega)$, $h|_{\partial \Omega} <0$ but $h>0$ somewhere in $\Omega$. Let $k > [h]_{0,1}$ and $u \in M^k$ be a solution of \[\int_\Omega a(\nabla u)\cdot \nabla(v-u) dx \geq 0 \; \forall v \in M^k\]
		Then \begin{enumerate}
			\item $u \geq 0$
			\item $u$ is unique
			\item $[u]_{0,1} \leq [h]_{0,1} < k$
		\end{enumerate}
	\end{theorem}
	\begin{theorem}
		Let $\Omega \subset \mathbb{R}^n$ be a bounded convex domain with $\partial \Omega \in C^1$. Let $h \in Lip(\Omega)$, $h|_{\partial \Omega} <0$ and $h>0$ somewhere in $\Omega$. Then $\exists! u \in Lip(\Omega)$, $u|_{\partial \Omega} = 0$ and $u \geq h$ in $\Omega$. \[F(u) \leq F(v) \forall v \in Lip(\Omega), \; v|_{\partial \Omega) = 0}\]
		where $F(u) = \int_\Omega \sqrt{1+\left|\nabla u\right|^2} \, dx$. Moreover, $u$ solves \[\int_\Omega \frac{\nabla u}{1+\left|\nabla u\right|^2} \nabla (v-u)\, dx \geq 0\] and $u$ satisfies $[u]_{0,1} \leq [h]_{0,1}$ and $||u||_{L^\infty} \leq [h]_{0,1} diam(\Omega)$.\\
		Moreover, define $I = \{x \in \Omega: u(x) = h(x)\}$ the coincidence set, then \[\int_\Omega \frac{\nabla u}{\sqrt{1+\left|\nabla u\right|^2}} \nabla \varphi \, dx = 0 \; \forall \varphi \in Lip(\Omega), \, supp(\varphi) \subset \Omega \setminus I\]
	\end{theorem}
	\begin{remark}
		Is $I\neq \varnothing$? Is it connected and what is its measure? 
	\end{remark}
	\section{The Schrödinger equation}
	We describe an electron in a universe with only one nucleus that attracts the electron. Motivated by the uncertainty principle of Heisenberg, the electron is described by a function $\psi: \mathbb{R}^n \to \mathbb{C}$, $||\psi||_{L^2(\mathbb{R}^n)}=1$ a wave function.\\
	Idea: $\Omega \subset \mathbb{R}^n$ open, $\int_\Omega \left|\psi(x)\right|^2 \, dx$ is the probability that the electron is in $\Omega$.\\
	We consider $n \geq 3$. Define \begin{itemize}
		\item \underline{Kinetic energy}: $\int_{\mathbb{R}^n} \left|\nabla \psi\right|^2 \, dx = T(\psi)$
		\item \underline{potential energy}: $V(\psi) = \int_{\mathbb{R}^n} v(x) \left|\psi(x)\right|^2 \, dx$. Typical choice would be $n=3$ (for obvious reasons), $v(x) = -\frac{z}{\left|x\right|}$ with a nucleus of charge $z$ at the origin.
		\item \underline{Total energy}: $E(\psi) = T(\psi) + V(\psi)$
	\end{itemize}
	\underline{Question:} Is \[E_0 = \inf\{E/\psi): \psi \in W^{1,2}(\mathbb{R}^n), \; ||\psi||_{L^2(\mathbb{R}^n)} = 1\}\] attained? $E_0$ is called the ground state energy. If a minimum exists, it is called ground state.
	\begin{definition}
		Let $f:\mathbb{R}^n \to \mathbb{C}$ then we say that $f$ \underline{vanishes at $\infty$} if $\forall a > 0$ \[\lambda^n(\{x \in \mathbb{R}^n: \left|f(x)\right| > a\}) < \infty\]
		We define \[D^1(\mathbb{R}^n) = \{f: \mathbb{R}^n \to \mathbb{C} \mid \nabla f \in L^2(\mathbb{R}^n), \text{ and $f$ vanishes at } \infty\}\]
	\end{definition}
	\begin{theorem}
		Let $n\geq 3$. Then $\exists S_n > 0$ s.t. $\forall f\in D^1(\mathbb{R}^n)$ \[S_n||f||_{L^\frac{2n}{n-2}(\mathbb{R}^n)} \leq ||\nabla f||_{L^2(\mathbb{R}^n)}\]
	\end{theorem}
	\begin{remark}
		\label{rem: potential}
		Let $n \geq 3$, $V \in L^{\frac{n}{2}} + L^\infty(\mathbb{R}^n)$. That means $v=v_1+v_2$ where $v_1 \in L^{\frac{n}{2}}$, $v_2 \in L^\infty(\mathbb{R}^n)$. Then $\exists \lambda \in \mathbb{R}$ s.t. $\forall \psi \in W^{1,2}(\mathbb{R}^n)$ with $||\psi||_{L^2} = 1$ \[E(\psi) \geq \frac{1}{2} \int_{\mathbb{R}^n} \left|\nabla \psi\right|^2 - \lambda - ||v_2||_{L^2} \geq -\lambda - ||v_2||_{L^2}\]
	\end{remark}
	Using this result, we can conclude that the potential energy is even continuous. That is \[\int_{\mathbb{R}^n} V(x)\left|\psi_j(x)\right|^2 \, dx \to \int_{\mathbb{R}^n} V(x) \left|\psi_0(x)\right|^2 \, dx\] for a convergent subsequence of a minimizing sequence $\psi_j$.
	\begin{lemma}
		If $f_k \rightharpoonup f$ in $W^{1,2}(\mathbb{R}^n)$, let $A\subset \mathbb{R}^n$ measurable s.t. $\lambda^n(A) < \infty$. Then \[\chi_A f_k \to \chi_A f \text{ in } L^{p}(\mathbb{R}^n)\] for $1\leq p \leq \frac{2n}{n-2}$.
	\end{lemma}
	\begin{lemma}
		Under the same assumptions as in \ref{rem: potential}, and the additional one that $v$ vanishes at $\infty$, i.e. $\forall a >0$ it holds that $\lambda^n(\{\left|V(x)\right|>a\}) < \infty$, then if $\psi_j \rightharpoonup \psi$ in $W^{1,2}(\mathbb{R}^n)$, $||\psi_j||_{L^2} = 1 \; \forall j$ then \[\int_{\mathbb{R}^n} V(x) \left|\psi_j(x)\right|^2 \, dx \to \int_{\mathbb{R}^n} V(x) \left|\psi(x)\right|^2 \, dx\]
	\end{lemma}
	\begin{theorem}[existence of the minimizer]
		Let $v \in L^{\frac{n}{2}} + L^\infty$ and let $v$ vanish at $\infty$. If $E_0 = \inf\{E(\psi) \mid \psi \in W^{1,2}(\mathbb{R}^n), ||\psi||_{L^2(\mathbb{R}^n)}\} <0$, then $\exists \psi_0 \in W^{1,2}(\mathbb{R}^n)$, $||\psi_0||=1$ s.t. $E(\psi_0) = E_0$. Moreover, $\psi_0$ is a weak solution of \[-\Delta \psi_0 + v\psi_0 = E_0\psi_0\] i.e. $\forall \varphi \in C_0^\infty(\mathbb{R}^n)$ \[\int_{\mathbb{R}^n} \nabla \psi_0 \nabla \varphi + v\psi_0 \varphi = E_0 \int_{\mathbb{R}^n} \psi_0 \varphi\]
	\end{theorem}
	\begin{remark}
		\begin{enumerate}
			\item $E_0$ in the Euler-Lagrange equation can be considered as a Lagrange-multiplier ensuring that the constraint $||\psi||_{L^2}=1$.
			\item We can write \[E_0 = \inf_{\substack{\psi \in W^{1,2}\\ \psi \neq 0}} \frac{E(\psi)}{||\psi||_{L^2}^2}\]
			\item The first condition is important to ensure existence of a minimizer. Indeed, if $v\equiv 0$ there is no ground state $\psi_0$.\\
			Let $u \in C_0^\infty(\mathbb{R}^n)$ s.t. $||u||_{L^1} = 1$. Then for $\lambda > 0$ $u_\lambda(x) = \lambda^{\frac{n}{2}}u(\lambda x)$ is a wave function. $$E(u\lambda) = T(u_\lambda)=\lambda^2T(u) \overset{\lambda \searrow 0}{\to} 0$$ Hence $E_0=0$ and $\not \exists \psi \in W^{1,2}(\mathbb{R}^n)$ with $||\psi||_{L^2}=1$ s.t. $E(\psi) = 0$. So a minimizer does not exist.
		\end{enumerate}
	\end{remark}
	\begin{remark}
		If $E_0 < 0$, one can look at excited states. With the same ideas, if \[E_1 = \inf\{E(\psi)\mid \psi \in W^{1,2}(\mathbb{R}^n), ||\psi||_{L^2} = 1, \langle\psi, \psi_0\rangle_{L^2} = 0\} <0\] there exists $\psi_1 \in W^{1,2}$, $||\psi_1||_{L^2} = 1$, $\langle\psi_1,\psi_0\rangle_{L^2} = 0$ s.t. \[E(\psi_1) = E_1\] and one can continue \[0 > E_k = \inf\{E(\psi) \mid \psi \in W^{1,2}, ||\psi||_{L^2}=1, \langle \psi, \psi_j \rangle_{L^2} = 0, \; 0 \leq j \leq k-1\}\]
		$\psi_k$ satisfies \[-\Delta \psi_k + v\psi_k = E_k\psi_k, \; E_k = E(\psi_k)\] and \[E_k = \inf_{\substack{\psi \in W^{1,2} \\ \psi \bot_{L^2} \psi_j \\ 0 \leq j \leq k-1}} \frac{E(\psi)}{||\psi||_{L^2}^2}\]
	\end{remark}
	\section{Parametric minimal surfaces (the plateau problem)}
	Let $B=B_1(0) \subset \mathbb{R}^2$ be the unit disk and \[\gamma: S^1 \to \mathbb{R}^m, \; m \geq 3\]
	a $C^2$ curve that is simple, i.e. $\gamma$ has no self intersection.\\
	\underline{Aim:} find a surface of disk type whose boundary is given by $\gamma$ and with minimal area.
	\subsection{Some basic concepts from differential geometry}
	\begin{definition}
		A surface $S \subset \mathbb{R}^m$, $m \geq 3$ is a 2-dimensional $C^l$-submanifold of $\mathbb{R}^m$ where $l\geq 1$, that is $\forall p \in S$ $\exists (U,F,V)$ where $U \subset \mathbb{R}^2$ open. $V\subset \mathbb{R}^m$ open, $p \in V\cap S$ and $F:U \to V\cap S$, $F \in C^2$ is a homeomorphismus and $\forall u \in U$ is $rank(DF(u)) = 2$.
	\end{definition}
	\begin{remark}
		The definition implies that locally the graph can be described by the graph of a $u:U \to \mathbb{R}^{m-2}$ and locally the surface can be described as a level set $S\cap V = \{x \in V: g(x)=0\}$.
	\end{remark}
	\begin{definition}[tangential space]
		Let $S$ be a surface in $\mathbb{R}^m$, $p \in S$, \[T_pS = \{v \in \mathbb{R}^m: \exists c: (-\varepsilon, \varepsilon) \to S \in C^1 \text{ s.t. } c(0)=p, C'(0) = v\}\] $T_pS$ is the tangential space to $S$ at $p$ with $\dim T_pS = 2$.
	\end{definition}
	If $(U,F,V)$ is a local chart at $p \in S$, then \[T_pS = span\left\{\frac{\partial F}{\partial u}(u,v), \frac{\partial F}{\partial v}(u,v)\right\}\] $U \subset \mathbb{R}^2$, $F U \to V \subset \mathbb{R}^m$.
	\underline{Aim:} define a scalar product in $T_pS$. The first fundamental form is \[g_p: T_pS \times T_pS \to \mathbb{R}\] with \[(u,v) \mapsto \langle u,v\rangle_{\mathbb{R}^m}\]
	If $(U,F,V)$ is a local chart at $p$, for $w_1,w_2 \in T_pS$, $w_1 = w_1^1 \frac{\partial F}{\partial u} + w_1^2 \frac{\partial F}{\partial V}$, $w_2 = w_2^1 \frac{\partial F}{\partial u} + w_2^2 \frac{\partial F}{\partial v}$ so that we can represent the first fundamental form with respect to this basis via the matrix $(g_{i,j}(u,v))^2_{i,j=1}$ by 
	\begin{itemize}
		\item $g_{1,1}(u,v) = \langle \frac{\partial F}{\partial u}(u,v), \frac{\partial F}{\partial v}(u,v)\rangle$, 
		\item $g_{1,2}(u,v) = g_{2,1}(u,v) = \langle \frac{\partial F}{\partial u}(u,v), \frac{\partial F}{\partial v}(u,v) \rangle$
		\item $g_{2,2}(u,v) = \langle \frac{\partial F}{\partial v}(u,v), \frac{\partial F}{\partial }(u,v) \rangle$.
	\end{itemize}
	We want to define for \[f: S \to \mathbb{R}\] continuous $\int_S f(x) dS(x)$\\
	\underline{First step}: Let $(U,F,V)$ be a local chart and consider $f$ such that $supp(f) \subset V \cap S$. \[\int_S f(x) dS(x) = \int_u f(F(u,v)) \sqrt{\det(g_{i,j}(u,v))^2_{i,j=1}} \, du \, dv\]
	\underline{Second step}: If $S$ is compact, hen $\exists$ finitely many charts $(U_i,F_i,V_i)_{i=1}$ s.t. $S \subset \bigcup_{i=^1}^N (V_i \cap S)$. Then consider $(\psi_i)_{i=1}^N$ a partition of unity associated to this cover. \[\psi_i \in C_0^\infty(\mathbb{R}^m; [0,1]), supp(\psi_i) \subset V_i, \; \forall i=1,...,N \text{ and } 1 = \sum_{i=1}^N \psi_i(x) \; \forall x \in S\]
	Then for $fS \to \mathbb{R}$ continuous, \[\int_S f(x) \, dS(x) = \sum_{i=1}^N \int_s (f\psi_i)(x) \; dS(x)\] If $S$ is described by only one chart, one can see that \[area(S) = \int_U \sqrt{\left|\frac{\partial F}{\partial u}\right|^2 \left|\frac{\partial F}{\partial v}\right|^2 - \left(\frac{\partial F}{\partial u} \cdot \frac{\partial F}{\partial v}\right)^2} \, d(u,v)\]
	\begin{remark}
		The area is well-defined for $F\in W^{1,2}(B,\mathbb{R}^m)$. 
	\end{remark}
	\begin{definition} \label{def: D(F)}
		\[D(F) = \frac{1}{2} \int_B \left|\frac{\partial F}{\partial u}\right|^2 + \left|\frac{\partial F}{\partial v}\right|^2 \; d(u,v)\] is called the Dirichlet energy of $F$.
	\end{definition}
	\subsection{Setup for the problems}
	Define for $\gamma: \partial B_1(0) \to \mathbb{R}^m$, $m \geq 3$, a closed and simple curve \[K(\gamma) = \{F \in W^{1,2} (B,\mathbb{R}^m): F|_{\partial B} \in C^0(\partial B, \mathbb{R}^m) \text{ is a monotone parametrization of } \gamma\}\]
	\underline{Question 1}: $\exists F \in K(\gamma)$ s.t. $A(F) \leq A(\tilde F)$ $\forall \tilde F \in K(\gamma)$? This is a difficult question.\\
	\underline{Question 2}: $\exists F \in K(\gamma)$ s.t. \[D(F) \leq D(\tilde F) \; \forall \tilde F \in K(\gamma)\] where $D(F)$ is defined in \ref{def: D(F)}.\\
	As we shall see, these two questions are equivalent.
	\begin{remark}
		$K(\gamma) \neq \varnothing$.
	\end{remark}
	\subsection{Euler-Lagrange equation for question 2}
	\begin{lemma}
		Let $F \in K(\gamma)$ be a minimizer of the Dirichlet energy in $K(\Gamma)$. Then $\Delta F = 0$ on $B$ i.e. $F$ is harmonic and $F \in C^\infty(B)$.
	\end{lemma}
	\begin{lemma}
		If $f \in L^1_{loc}(\Omega)$ such that $\int_\Omega f \Delta \varphi \, dx = 0$ $\forall \varphi \in C_0^\infty(\Omega)$. Then $f \in C^\infty(\Omega)$ and $\Delta f = 0$.
	\end{lemma}
	As an abbreviation for the integral making up the Dirichlet-energy $D(F)$ we write $\left|DF\right|^2$.
	\begin{remark}
		Let $X: B_1(0) \to \mathbb{R}^m$ s.t. $\Delta X = 0$. Let $(u_1,u_2) \in B_1(0)$. Then the map \[z = u_1 + iu_2 \mapsto g(z) = \left|\partial_{u_1} X\right|^2 - \left|\partial_{u_2}\right|^2 - 2i \partial_{u_1} X \partial_{u_2}X\] is holomorphic. Indeed by the Cauchy-Riemann equations, $g$ is holomorphic $\iff$ $\partial_{\overline{z}} g = 0$ where \[\partial_{\overline{z}} g = \frac{1}{2}\left(\partial_{u_1}g + i \partial_{u_2}g\right)\]  
		It is easily verifiable that $\partial_{\overline{z}} g = 4 \partial_z X \cdot \partial_{\overline{z}} \partial_z X = \partial_z X \Delta X$.
	\end{remark}
	\begin{theorem}
		Let $F \in K(\gamma)$ be a critical point of the Dirichlet-energy in $K(\gamma)$. By that we mean that \begin{enumerate}
			\item $\frac{d}{dt} D(F+t\varphi)|_{t=0} = 0$ $\forall \varphi \in C_0^\infty(B;\mathbb{R}^m)$
			\item $\frac{d}{dt} D(F\circ \varphi^{-1}_t, B_t)|_{t=0} = 0$ where $(\varphi_t)_t$ is a family of diffeomorphisms. $\varphi_t = Id + t\psi$ with $\psi \in C^2(\overline{B_1(0)})$ and $\varphi_t: B_1(0) \to B_t = \varphi(B_1(0))$.
		\end{enumerate}
		Then $F$ solves the following problem:
		\begin{itemize}
			\item $\Delta F = 0$ in $B_1(0)$
			\item $F|_{\partial B_1(0)}$ is a monotone parametrization of $\gamma(\partial B_1(0))$
			\item $\left|\partial_{u_1} F\right| = \left|\partial_{u_2} F\right|$
			\item $\langle \partial_{u_1} F, \partial_{u_2} F \rangle = 0$
		\end{itemize}
		The last two properties yield that $F$ is a so-called conformal parametrization.
	\end{theorem}
	\begin{remark}
		If $X$ minimizes the Dirichlet-energy, then $X$ satisfies the second point in the previous theorem.
	\end{remark}
	\begin{lemma}
		If $u \in C^1_{b'}(B;\mathbb{R})$, $r \in (0,1)$ \[u_r: \partial B_1(0) \to \mathbb{R}\] $u_r(e^{i\varphi}) = u(re^{i\varphi})$. Then $\exists u_{bd}: \partial B_1(0) \to \mathbb{R}$ s.t. \[u_r \to u_{bd}\] in $L^2(\partial B)$.
	\end{lemma}
	\begin{definition}
		Define $K^*(\gamma) = \{F\in C^1_{b'}(B;\mathbb{R}^m)\}$.
	\end{definition}
	Let $RF \in C^0(\partial B_1(0); \mathbb{R}^m)$ s.t. RF is a weak monotone parametrization of $\Gamma \circ \gamma(\partial B_1(0))$.\\
	IDEA: Take $(\tilde{F_k})_{k \in \mathbb{N}} \subset K^+(\gamma)$ be a minimizing sequence, $R\tilde{F_k} \in C^0(\partial B_1(0); \mathbb{R}^m)$. We take $F_k$ s.t. \[-\Delta F_k = 0 \text{ in } B_1(0)\]
	\[F_k = R\tilde{F_k} \text{ on } \partial B_1(0)\]
	\subsection{The Dirichlet problem on the disk}
	LEt $B = B_1(0) \subset \mathbb{R}^2$ and $\varphi: \partial B_1(0)  \to \mathbb{R}$. We study the problem \[\begin{cases}
		-\Delta u = 0 & \text{ in } B\\
		u = \varphi & \text{ on } \partial B
	\end{cases}\]
	Note that $\Delta u = \sum_{i=1}^{n} \frac{\partial^2 u}{\partial x_j^2}$.
	\begin{lemma}
		If $u \in C^0(\overline{B})$ is harmonic in $B$, then \[u(x) = \frac{1}{2\pi} (1-||x||^2) \int_{\partial B} \frac{u(y)}{\left|x-y\right|^2} d_S(y)\] for $x\in B$.
	\end{lemma}
	This suggests to consider \[E: L^1(\partial B) \to C^\infty(B)\] where \[\varphi \mapsto (E_\varphi)(x) = \frac{1}{2\pi}(1-||x||^2)\int_{\partial B} \frac{\varphi(y)}{\left|x-y\right|^2} \, ds(y)\] with $x\in B$. We call this the harmonic extension.
	\begin{theorem}
		For $\varphi \in C^0(\partial B)$, $\forall x_0 \in \partial B$ $$\exists \lim_{\substack{x \to x_0 \\ x \in B}} (E_\varphi)(x) = \varphi(x)$$ Hence $E|_{C^0(\partial B)}: C^0(\partial B) \to C^\infty(B) \cap C^0(\overline{B})$.
	\end{theorem}
	\begin{theorem}
		Let $v \in C^1_{b'}(B)$ and $v_{bd} \in L^2(\partial B_1(0))$ its restriction to the boundary as before. Let $h = E(v_{bd})$ be the harmonic extension of $v_{bd}$ Then \[D(h) \leq D(v)\]
	\end{theorem}
	\begin{theorem}[Maximum principle]
		Let $u \in C^0(\overline{B})$ be harmonic in $B$. Then \[\max_{x \in \overline{B}} u(x) = \max_{x \in \partial B} u(x)\] The same holds for $\min$.
	\end{theorem}
	\begin{theorem}[Poisson formula]
		If $u$ is harmonic in $B$ and $K\subset B$ compact, then $\exists d > 0$ s.t. \[\sup_{K} \left|D^\alpha_u\right| \leq \left(\frac{2\left|\alpha\right|}{d}\right)^{\left|\alpha\right|} \sup_B u\] In particular, the uniform limit of harmonic functions is harmonic.
	\end{theorem}
	\section{Solution of the Plateau Problem}
	Recall: We want to minimize $D(F) = \frac{1}{2}\int_B \left|D_x F\right|^2 + \left|D_y F\right|^2 d(x,y)$ for $F \in K^*(y)$.\\
	\begin{lemma}
		Let $\Phi \in Aut(B)$, $f \in W^{1,2}(B_1(0))$ then $D(f) = D(f\circ \Phi)$.
	\end{lemma}
	\begin{definition}
		Let $P_j = \gamma(e^{i(j-1)\frac{2\pi}{3}})$ for $j=1,2,3$.
	\end{definition}
	\begin{lemma}
		Let $f \in C^0(\overline{B}, \mathbb{R}^m)\cap C^1_{b'}(B, \mathbb{R}^m)$, $w_0 \in \overline{B}$ and $(r,\varphi)$ the polar coordinates centred at $w_0$. On $B_2(w_0) \cap B$ let \[\tilde{f}(r,\varphi) = f(w_0 + re^{i\varphi})\] Then $\forall \delta \in (0,1)$ $\exists \varrho \in [\delta, \sqrt{\delta}]$ s.t. on \[C_\varrho = \partial B_\varrho(w_0) \cap B\] we have \[\int_{C_\varrho} \left|\partial_\varphi \tilde{f}\right|^2 ds(\varphi) \leq \frac{4\varrho}{\left|\log \varphi\right|} D(f)\]
	\end{lemma}
	\begin{theorem}
		Let $\Gamma \subset \mathbb{R}^m$ be a closed simple curve, \[\Gamma = \gamma(\partial B_1(0))\] where $\gamma: \partial(B_1(0)) \to \mathbb{R}^m$ s.t. \[K^*(\gamma) = \{F \in C^1_{b'}(B_1(0),\mathbb{R}^m) : RF \in C^0(\partial B_1(0), \mathbb{R}^m) \text{ a weak monotone partition of } \Gamma\} \neq \varnothing\]
		Then $\exists F \in K^*(\gamma)$ s.t. $D(F) \leq D(\tilde{F})$ $\forall \tilde{F} \in K^*(\gamma)$.
	\end{theorem}
	\begin{theorem}
		It holds that $A(X) \leq D(X)$ and equality holds iff $X$ is a conformal parametrization. Further, minimizers of the Dirichlet energy are harmonic and conformal parametrizations. 
	\end{theorem}
	\begin{theorem}
		Let $\Gamma = \gamma(\partial B_1(0))$ be a closed simple curve, $\gamma \in C^2$ s.t. $K^*(\gamma) \neq \varnothing$. Then $\exists F \in K^*(\gamma)$ s.t. \[D(F)\leq D(\tilde{F}) \; \forall \tilde{F} \in K^*(\gamma)\]
	\end{theorem}
	\begin{theorem}
		Let $F \in W^{1,2}(B_1(0), \mathbb{R}^m)\cap C^0(\overline(B_1(0)), \mathbb{R}^m)$. Then $\forall \varepsilon > 0 \; \exists \tau_\varepsilon : \overline{B_1(0)} \to \overline{B_1(0)}$ homeomorphism s.t. $\tau_\varepsilon \in W^{1,2}$, $$F \circ \tau_\varepsilon \in W^{1,2}(B_1(0), \mathbb{R}^m) \cap C^0(\overline{B_1(0)}, \mathbb{R}^m)$$ and \[D(F \circ \tau_\varepsilon) \leq A(F\circ \tau_\varepsilon) + \varepsilon\]
	\end{theorem}
	\begin{corollary}[Minimizer of the area functional]
		Under the assumptions of the previous two theorems \[\inf_{F \in K^*(\gamma)} A(F) = \inf_{F \in K^*(\gamma)} D(F)\] and the infima are attained.
	\end{corollary}
\section{A reaction-diffusion equation}
	Let $p>1$, $\Omega \subset \mathbb{R}^n$ a $C^1$-domain, $n \geq 2$ and $\lambda \in \mathbb{R}$. We consider \[\begin{cases}
		-\Delta u = \lambda u + \left|u\right|^{p-1} u, & \text{ in } \Omega\\
		u = 0, & \text{ on } \partial \Omega
	\end{cases}\]
	$u \in W_0^{1,2}(\Omega) \left( = \overline{C_0^\infty(\Omega)}^{||\cdot||_{W^{1,2}}}\right)$ is called a weak solution to this differential equation if \[\int_\Omega \nabla u \nabla \varphi \, dx = \lambda \int_\Omega u \varphi + \int_\Omega \left|u\right|^{p-1} u \varphi \; \forall \varphi \in C_0^\infty(\Omega)\]
	\begin{remark}
		$u\equiv 0$ is a weak solution. For sufficiently small $\lambda$ there are other weak solutions.
	\end{remark}
\section{The mountain pass problem}
	\subsection{The Palais-Smale Condition}
	Let $H$ be a Hilbert space on $\mathbb{R}$. We denote the scalar product by $\langle \cdot , \cdot \rangle$. We use the usual definition of Fréchet-differentiability.
	\begin{definition}
		Let $E \in C^1(H;\mathbb{R})$. If $E$ is  Fréchet-differentiable at $u$ $\forall u \in H$ and the map \[\begin{cases}
			E': H \to L(H; \mathbb{R})\\
			u \mapsto E'(u)
		\end{cases}\]
	is continuous 
	\end{definition}	
	\begin{definition}
		Let $E \in C^1(H;\mathbb{R})$. \begin{enumerate}
			\item A sequence $(u_k)_{k \in \mathbb{N}} \subset H$ is called a Palais-Smale sequence for $E$ if \begin{itemize}
				\item $\exists \lim_{k \to \infty} E(u_k)$
				\item $\exists \lim_{k \to \infty} \left|\left|\nabla E(u_k)\right|\right|_H = 0$ 
			\end{itemize}
		\item $E$ satisfies a Palais-Smale condition if every Palais-Smale sequence admits a convergent subsequence (w.r.t. the norm in $H$).
		\end{enumerate}
	\end{definition}
	\begin{definition}
		Let $E \in C^1(H;\mathbb{R})$. For $\beta \in \mathbb{R}$, $\delta > 0$, $\varrho > 0$, define \[K_\beta = \{u \in H \mid E(u) = \beta \land \nabla E(u) = 0\}\]
		\[E_\beta = \{u \in H \mid E(u) < \beta\}\]
		\[N_{\beta, s} = \{u \in H \mid \left|E(u) - \beta\right| < \delta \land \left|\left|\nabla E(u)\right|\right| < \delta\}\]
		\[U_{\beta, \varrho} = \{u \in H \mid \exists r \in K_\beta : \; \left|\left|u-v\right|\right|_H < \varrho\} = \bigcup_{u \in K_\beta} B_\varrho(u)\]
	\end{definition}
	\begin{lemma}
		Let $E \in C^1(H;\mathbb{R})$ that satisfies the Palais-Smale condition. Then for $\beta \in \mathbb{R}$ \begin{enumerate}
			\item $K_\beta$ is compact
			\item $(N_{\beta,\delta})_{\delta > 0}$ is a neighbourhood basis of $k_\beta$, i.e. Let $U$ open s.t. $U\supset K_\beta$ then $\exists \delta > 0$ s.t. $U\supset N_{\beta, \delta}$
			\item $(U_{\beta, \varrho})_{\varrho > 0}$ is a neighbourhood basis of $K_\beta$.
		\end{enumerate}
	\end{lemma}
	\subsection{A deformation Lemma in $H$}
	PROBLEM: In $\mathbb{N}^n$, we could use uniform continuity on compact sets. This no longer holds.\\
	SOLUTION I: Take a deformation $\frac{\partial}{\partial t} \Phi(x,t) \approx -\nabla E(\Phi(x,t))$. This however requires $E \in C^{1,1}$.\\
	SOLUTION II: This works for $e \in C^1$. 
	\begin{definition}
		Let $X$ be a topological space. \begin{enumerate}
			\item a covering $(U_i)_{i \in I}$ is a refinement of a cover $(V_j)_{j \in J}$ if $\forall i \in I$ $\exists j \in J$ such that $U_i \subset V_j$.
			\item a covering of $X$ is called locally finite if $\forall x \in X$ $\exists U$ neighbourhood of $X$ such that $U \cap U_i \neq \varnothing$ for finitely many $i$.
			\item $X$ is called para-compact if $X$ is Hausdorff and to any open covering $X$ there exists a locally finite refinement.
		\end{enumerate}
	\end{definition}
	\begin{theorem}
		Every metrizable topological space is paracompact.
	\end{theorem}
	\begin{definition}
		Let $E \in C^1(H;\mathbb{R})$, $\tilde{H} = \{u \in H \mid \nabla E(u) \neq 0\}$. Then $G: \tilde{H} \to H$ is a locally Lipschitz map is pseudogradient for $E$ if $\forall u \in \tilde{H}$ it holds that \[\left|\left|G(u)\right|\right|_h < 2 \min\{1,\left|\left|\nabla E(u)\right|\right|_H\}\]
		\[\langle G(u), \nabla E(u)\rangle > \frac{1}{2} \min\{1,\left|\left|\nabla E(u)\right|\right|_H\} \left|\left|\nabla E(u)\right|\right|\]
	\end{definition}
	\begin{theorem}
		Let $E \in C^1(H;\mathbb{R}$), $\tilde{H}$ as above, then $E$ admits a pseudogradient $G: \tilde{H} \to H$
	\end{theorem}
	\begin{theorem}
		Let $E \in C^1(H)$ satisfy the PS condition. Let $\beta \in \mathbb{R}$, $\varepsilon_0 > 0$ and $N$ be a neighbourhood of $K_\beta = \{u \in H \mid E(u) = \beta \land \nabla E(u) = 0\}$ (note that $K_\beta = \varnothing, N = \varnothing$ are fine as well).\\
		The $\exists \varepsilon \in (0,\varepsilon_0)$ and a continuous family of homeomorphisms $\Phi(t,\cdot): H \to H$, $t \in \mathbb{R}$ such that \begin{enumerate}
			\item $\Phi(t,u) = u$ if $t=0$ or $\nabla E(u) = 0$ or $\left|E(u) - \beta\right| \geq \varepsilon_0$.
			\item $\forall v \in H$ it holds that $t \mapsto E(\Phi(t,u))$ is non-increasing
			\item $\Phi(1,E_{\beta+\varepsilon}\setminus N) \subset E_{\beta - \varepsilon}$ and $\Phi(1,E_{\beta + \varepsilon}) \subset E_{\beta - \varepsilon} \cup N$ where $E_\alpha = \{u \in H \mid E(u) < \alpha\}$
		\end{enumerate}
		Moreover \[\Phi(t,\cdot) \circ \Phi(s,\cdot) = \Phi(t+s, \cdot)\]
	\end{theorem}
	\subsection{Mountain Pass Theorem}
	\begin{theorem}
		Let $H$ be a Hilbert space, $E \in C^1(H)$ satisfy the PS condition. Moreover, assume \begin{enumerate}
			\item $E(0)=0$
			\item $\exists \varrho>0, \exists \alpha > 0$ s.t. \[||u||_H = \varrho \Rightarrow E(u) \geq \alpha\]
			\item $\exists u_1 \in H$, $||u_1||_H > \varrho$ s.t. $E(u_1) < \alpha$
		\end{enumerate}
		Let $P = \{p \in C^0([0,1];H)$ s.t. $p(0) = 0, p(1) = u_1\}$ and $\beta = \inf_{p \in P}\sup_{u \in p} E(u) \geq \alpha$. Then $\beta$ is a critical value of $E$, i.e. $\exists u_{crit}$ critical point of $E$ s.t. $E(u_{crit}) = \beta$.
	\end{theorem}
	\subsection{An application}
	Let $\Omega \subset \mathbb{R}^n$ be a bounded, smooth domain, $n \geq 3$. $g: \Omega \times \mathbb{R} \to \mathbb{R}$ be a continuous function s.t. \[\exists c > 0, p \in (1,\frac{n+2}{n-2}): \; \left|g(x,t)\right| \leq c (1+\left|t\right|^p)\]
	The aim is to show existence of a non-trivial weak solution to \[\begin{cases}
		-\Delta u = g(x,u) & \text{ in } \Omega\\
		u = 0 & \text{ on } \partial \Omega
	\end{cases}\]
	If $g(x,0) = 0$ $\forall x$, then 0 is a weak solution.
	Let $G: \Omega\times \mathbb{R} \to \mathbb{R}$ such that \[G(x,t) = \int_0^t g(x,s) ds\] and $e: W^{1,2}_0(\Omega) \to \mathbb{R}$ s.t. \[E(u) = \frac{1}{2}\int_\Omega \left|\nabla u\right|^2 - \int_\Omega G(x,u) dx\]
	\begin{lemma}
		$E \in C^1(W^{1,2}_0(\Omega);\mathbb{R})$ with \[DE(u)(\varphi) = \int_\Omega \nabla u \cdot \nabla \varphi - \int_\Omega g(x,u)\varphi dx\] for $u,\varphi \in W^{1,2}_0(\Omega)$.
	\end{lemma}
	\begin{lemma}
		If $\exists R_0 > 0$, $\exists q > 2$ s.t. $\forall x \in \Omega$ and $\forall \left|t\right| \geq R_0$ \[0 < q G(x,t) \leq g(x,t)t\] (i.e. $g$ ahs super-linear growth at $\infty$) then $E$ satisfies a PS condition.
	\end{lemma}
	\begin{theorem}
		If $f_k \to f$ almost everywhere, $f_k \in L^r(\Omega)$, $1 \leq r < \infty$, $\left|\Omega\right|<\infty$, then the following are equivalent
		\begin{enumerate}
			\item $f_k \to f$ in $L^r(\Omega)$, $f \in L^r(\Omega)$
			\item uniform equicontinuity of the $L^r$ norm. That is \[\sup_k \int_E \left|f_k\right|^r \, dx \to 0 \text{ for } \left|E\right| \to 0\]
		\end{enumerate}
	\end{theorem}
	\begin{theorem}
		Additionally, assume that $\forall \varepsilon>0 \exists \delta>0$ s-t- \[\frac{g(x,t)}{t} < \varepsilon \; \forall x \in \Omega \; \forall \left|t\right| < \delta\] then the PDE \[\begin{cases}
			- \Delta u = g(x,u) \text{ in } \Omega\\
			u = 0 \text{ on } \partial \Omega
		\end{cases}\] admits a non-trivial weak solution.
	\end{theorem}
\section{$\Gamma$-convergence}
	\underline{First motivation:} Consider a family of problems \[\inf\{F_j(u) \mid u \in X_j\}, j \in \mathbb{N}\]
	where $X_j = (X_j,d_j)$ is a separable metric space, \[F_j : X_j \to \mathbb{R}\] are functionals.\\
	Question: Is there a separable metric space $X$ and $F:X \to \mathbb{R}$ s.t. $\inf\{F(u) \mid u \in X\}$ is an approximation of $\inf\{F_j(u) \mid u \in X_j\}$ for $j$ big enough?\\
	\underline{Second motivation:} Consider a double well potential $W: \mathbb{R} \to [0,\infty)$, that is $W$ has two distinct positions $\alpha$ and $\beta$ where the potential is zero. A state with zero energy can jump infinitely often from $\alpha$ to $\beta$. Here $W(\alpha) = W(\beta) = 0$, $W(x) > 0 \; \forall x \notin \{\alpha, \beta\}$, $W \in C^0$ with $\lim_{x \to \infty} W(x) = \infty$. However, the solution in experiments does not have that many jumps.\\
	IDEA: Approximate the physical solution with problems where we penalize the jumps.
\subsection{Definitions}
	Note that the concept of convergence we use on the space $X$ is fundamental to get $\Gamma$-convergence. For instance, we will see for the Modica-Mortola-functional that the energy is bounded in $W^{1,2}$ bit $\Gamma$-convergence will be in $L^1$.
	\begin{remark}
		What do we do, when each $F_j$ is defined in another metric space?\\
		Solution: Take $X$ big enough such that $X \supset X_j \; \forall j$, $X$ a metric space and consider $\tilde{F}_j : X \to \mathbb{R}$ where \[\tilde{F}_j(x) = \begin{cases}
			F_j(x), & x \in X_j\\
			\infty, & \text{ otherwise}
		\end{cases}\]
	\end{remark}
	$\Gamma$-convergence was developed to study minimization problems. That is our aim is the following:\\
	If $x_j$ minimizes $F_j$ and $x_j \to x$ in $X$ and $F_j$ $\Gamma$-converges to $G$ we would get that $x$ is a minimizer of $F$.\\
	In the following, $(X,d)$ is a separable metric space.
	\begin{definition}
		Let $F_n)_{n \in \mathbb{N}}$ be a sequence of energy functionals $F_n: X \to [-\infty,\infty]$. We say that $(F_n)_{n \in \mathbb{N}}$ $\Gamma$-converges to $F: X \to [\infty,\infty]$ and we write $F = \Gamma-\lim_{n \to \infty} F_n$ in $X$ if \begin{enumerate}
			\item \underline{$\liminf$ inequality:} $\forall x \in X,$ $\forall (x_n)_{n \in \mathbb{N}} \subset X$ s.t. $x_n \to x$ in $X$ it holds that \[F(X) \leq \liminf_{n \to \infty} F_n(x_n)\]
			\item \underline{$\limsup$-inequality:} $\forall x \in X$ $\exists (x_n)_{n \in \mathbb{N}}$ s.t $x_n \to x$ in $X$ and \[F(X) \geq \limsup_{n \to \infty} F_n(x_n)\] 
		\end{enumerate}
		$F$ is called the $\Gamma$-limit of $F_n$.
	\end{definition}
	\begin{remark}
		\begin{itemize}
			\item If LII is satisfied, then LSI can be written as \[\forall x \in X \; \exists x_n \to x \; \text{ in } X \text{ s.t. } F(x) = \lim_{n\to \infty} F_n(x_n)\]
			\item The definition of $\Gamma$-convergence is not symmetric. As a consequence, in general \[\Gamma-\lim_{j \to \infty} -F_j \neq -\Gamma-\lim_{j \to \infty} F_j\]
		\end{itemize}
	\end{remark}
\subsection{Examples}
	\begin{enumerate}
		\item Consider $f_j: \mathbb{R} \to \mathbb{R}$ where \[f_j(x) = \begin{cases}
			1, & x = 0\\
			0, & \text{ otherwise}
		\end{cases}, \; j \in \mathbb{N}\]
		It holds that $\Gamma-\lim_{j \to \infty} f_j \equiv 0$.
		\item Let $f: X \to \mathbb{R}$ and $f_j = f$ $\forall j \in \mathbb{N}$, i.e. a constant sequence. When is $f = \Gamma\lim_{j \to \infty} f_j$ in $X$?\\
		By the LII it is necessary that $\forall x \in X, \forall x_j \to x$ in $X$ that \[f(x) \leq \liminf_{j \to \infty} f_j(x_j) = \liminf_{j \to \infty}f(x_j)\] that is $f$ should be lower semicontinuous. This is actually also sufficient.
		\item If $f_j \to f$ uniformly and $f$ is lower semicontinuous then $f = \Gamma-\lim_{j \to \infty} f_j$.
		\item $f_j,g_j: \mathbb{R} \to \mathbb{R}$ where $f_j(x) = -\cos(j\cdot x)$ and $g_j(x) = \cos(j \cdot x)$. Then $\Gamma-\lim_{j \to \infty}f_j = \Gamma-\lim_{j \to \infty} g_j = -1$.
		\item If $f  \Gamma-\lim_{j \to \infty} f_j$ and $g$ is continuous, then \[f+g = \Gamma-\lim_{j \to \infty} (f_j+g)\]
	\end{enumerate}
	\subsection{Application}
	Let $W: \mathbb{R} \to [0, \infty)$ be a $C^1$ function s.t. \[W(x) = 0 \Leftrightarrow x \in \{\alpha, \beta\}\]
	with $\alpha < \beta$. For $\varepsilon > 0$ consider \[F_\varepsilon: W^{1,2}(\Omega) \to \mathbb{R}\]\[F_\varepsilon(u) = \frac{1}{\varepsilon} \int_\Omega W(u(x)) \; dx + \varepsilon \int_\Omega \left|\nabla u\right|^2 \; dx\]
	We are interested in the $\Gamma-\lim$ for $\varepsilon \to 0$.
	\begin{theorem}
		Let $\varepsilon_n \searrow 0$ and $(u_n)_{n \in \mathbb{N}} \subset W^{1,2}(\Omega)$ be such that \[\sup_{n \in \mathbb{N}} F_{\varepsilon_n} (un) < \infty\] Then there exists a subsequence $(u_{n_k})_{k \in \mathbb{N}}$ and there exists a $u \in BV(\Omega, \{\alpha, \beta\})$ such that \[u_{n_k} \to u \;(k \to \infty) \text{ in } L^1 \]
	\end{theorem}
	We now extend the functional $F_\varepsilon$ as follows \[F_\varepsilon: L^1(\Omega) \to (-\infty, \infty]\] by \[F_\varepsilon(u) = \begin{cases}
		\frac{1}{\varepsilon} \int_\Omega W(u) + \varepsilon \int_\Omega \left|\nabla u\right|^2, & u \in W^{1,2}(\Omega \land \int u = m\\
		\infty, & \text{ otherwise})
	\end{cases}\] 
	AIM: $F_\varepsilon$ $\Gamma$-converges in $L^1(\Omega)$ to \[F(u) = \begin{cases}
		c_W \times \left|\{\text{ jumps of $u$}\}\right|, & u \in BV(\Omega, \{\alpha, \beta\} \land \int_\Omega u = m\\
		\infty, & \text{ otherwise})
	\end{cases}\]
\end{document}